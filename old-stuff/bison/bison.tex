\input texinfo @c -*-texinfo-*-
@comment %**start of header
@setfilename bison.info
@settitle Bison Reference Manual
@setchapternewpage odd
@synindex fn cp
@synindex vr cp
@comment %**end of header

@ifinfo
This file documents the Bison parser generator.

Copyright (C) 1988, 1989 Free Software Foundation, Inc.

Permission is granted to make and distribute verbatim copies of
this manual provided the copyright notice and this permission notice
are preserved on all copies.

@ignore
Permission is granted to process this file through Tex and print the
results, provided the printed document carries copying permission
notice identical to this one except for the removal of this paragraph
(this paragraph not being relevant to the printed manual).

@end ignore
Permission is granted to copy and distribute modified versions of this
manual under the conditions for verbatim copying, provided also that the
sections entitled ``GNU General Public License'' and ``Conditions for
Using Bison'' are included exactly as in the original, and provided that
the entire resulting derived work is distributed under the terms of a
permission notice identical to this one.

Permission is granted to copy and distribute translations of this manual
into another language, under the above conditions for modified versions,
except that the sections entitled ``GNU General Public License'',
``Conditions for Using Bison'' and this permission notice may be
included in translations approved by the Free Software Foundation
instead of in the original English.
@end ifinfo

@titlepage
@sp12
@center @titlefont{BISON}
@sp2
@center The YACC-compatible Parser Generator
@sp2
@center 12 October 1988
@sp2
@center by Charles Donnelly and Richard Stallman
@page
@vskip 0pt plus 1filll
Copyright @copyright{} 1988 Free Software Foundation

Permission is granted to make and distribute verbatim copies of
this manual provided the copyright notice and this permission notice
are preserved on all copies.

@ignore
Permission is granted to process this file through TeX and print the
results, provided the printed document carries copying permission
notice identical to this one except for the removal of this paragraph
(this paragraph not being relevant to the printed manual).

@end ignore
Permission is granted to copy and distribute modified versions of this
manual under the conditions for verbatim copying, provided also that the
sections entitled ``GNU General Public License'' and ``Conditions for
Using Bison'' are included exactly as in the original, and provided that
the entire resulting derived work is distributed under the terms of a
permission notice identical to this one.

Permission is granted to copy and distribute translations of this manual
into another language, under the above conditions for modified versions,
<except that the sections entitled ``GNU General Public License'',
``Conditions for Using Bison'' and this permission notice may be
included in translations approved by the Free Software Foundation
instead of in the original English.
@end titlepage
@page

@node Top, Introduction, (DIR), (DIR)

@menu
* Introduction::
* Conditions::
* Copying::           The GNU General Public License says
                        how you can copy and share Bison

Tutorial sections:
* Concepts::          Basic concepts for understanding Bison.
* Examples::          Three simple explained examples of using Bison.

Reference sections:
* Grammar File::      Writing Bison declarations and rules.
* Interface::         C-language interface to the parser function @code{yyparse}.
* Algorithm::         How the Bison parser works at run-time.
* Error Recovery::    Writing rules for error recovery.
* Context Dependency::What to do if your language syntax is too
			messy for Bison to handle straightforwardly.
* Debugging::         Debugging Bison parsers that parse wrong.
* Invocation::        How to run Bison (to produce the parser source file).
* Table of Symbols::  All the keywords of the Bison language are explained.
* Glossary::          Basic concepts are explained.
* Index::             Cross-references to the text.
@end menu

@node Introduction, Conditions, Top, Top
@unnumbered Introduction
@cindex introduction

@dfn{Bison} is a general-purpose parser generator which converts a grammar
description into a C program to parse that grammar.  Once you are
proficient with Bison, you may use it to develop a wide range of language
parsers, from those used in simple desk calculators to complex programming
languages.

Bison is upward compatible with Yacc: all properly-written Yacc grammars
ought to work with Bison with no change.  Anyone familiar with Yacc
should be able to use Bison with little trouble.  You need to be fluent in
C programming in order to use Bison or to understand this manual.

We begin with tutorial chapters that explain the basic concepts of using
Bison and show three explained examples, each building on the last.  If you
don't know Bison or Yacc, start by reading these chapters.  Reference
chapters follow which describe specific aspects of Bison in detail.

Bison was basically written by Robert Corbett, and made Yacc-compatible
by Richard Stallman.

@node Conditions, Copying, Introduction, Top
@unnumbered Conditions for Using Bison

Bison grammars can be used only in programs that are free software.  This
is in contrast to what happens with the GNU C compiler and the other
GNU programming tools.

The reason Bison is special is that the output of the Bison utility---the
Bison parser file---contains a verbatim copy of a sizable piece of Bison,
which is the code for the @code{yyparse} function.  (The actions from your
grammar are inserted into this function at one point, but the rest of the
function is not changed.)

As a result, the Bison parser file is covered by the same copying
conditions that cover Bison itself and the rest of the GNU system: any
program containing it has to be distributed under the standard GNU copying
conditions.

Occasionally people who would like to use Bison to develop proprietary
programs complain about this.

We don't particularly sympathize with their complaints.  The purpose of the
GNU project is to promote the right to share software and the practice of
sharing software; it is a means of changing society.  The people who
complain are planning to be uncooperative toward the rest of the world; why
should they deserve our help in doing so?

However, it's possible that a change in these conditions might encourage
computer companies to use and distribute the GNU system.  If so, then we
might decide to change the terms on @code{yyparse} as a matter of the
strategy of promoting the right to share.  Such a change would be
irrevocable.  Since we stand by the copying permissions we have announced,
we cannot withdraw them once given.

We mustn't make an irrevocable change hastily.  We have to wait until there
is a complete GNU system and there has been time to learn how this issue
affects its reception.

@node Copying, Concepts, Conditions, Top
@unnumbered GNU GENERAL PUBLIC LICENSE
@center Version 1, February 1989

@display
Copyright @copyright{} 1989 Free Software Foundation, Inc.
675 Mass Ave, Cambridge, MA 02139, USA

Everyone is permitted to copy and distribute verbatim copies
of this license document, but changing it is not allowed.
@end display

@unnumberedsec Preamble

  The license agreements of most software companies try to keep users
at the mercy of those companies.  By contrast, our General Public
License is intended to guarantee your freedom to share and change free
software---to make sure the software is free for all its users.  The
General Public License applies to the Free Software Foundation's
software and to any other program whose authors commit to using it.
You can use it for your programs, too.

  When we speak of free software, we are referring to freedom, not
price.  Specifically, the General Public License is designed to make
sure that you have the freedom to give away or sell copies of free
software, that you receive source code or can get it if you want it,
that you can change the software or use pieces of it in new free
programs; and that you know you can do these things.

  To protect your rights, we need to make restrictions that forbid
anyone to deny you these rights or to ask you to surrender the rights.
These restrictions translate to certain responsibilities for you if you
distribute copies of the software, or if you modify it.

  For example, if you distribute copies of a such a program, whether
gratis or for a fee, you must give the recipients all the rights that
you have.  You must make sure that they, too, receive or can get the
source code.  And you must tell them their rights.

  We protect your rights with two steps: (1) copyright the software, and
(2) offer you this license which gives you legal permission to copy,
distribute and/or modify the software.

  Also, for each author's protection and ours, we want to make certain
that everyone understands that there is no warranty for this free
software.  If the software is modified by someone else and passed on, we
want its recipients to know that what they have is not the original, so
that any problems introduced by others will not reflect on the original
authors' reputations.

  The precise terms and conditions for copying, distribution and
modification follow.

@iftex
@unnumberedsec TERMS AND CONDITIONS
@end iftex
@ifinfo
@center TERMS AND CONDITIONS
@end ifinfo

@enumerate
@item
This License Agreement applies to any program or other work which
contains a notice placed by the copyright holder saying it may be
distributed under the terms of this General Public License.  The
``Program'', below, refers to any such program or work, and a ``work based
on the Program'' means either the Program or any work containing the
Program or a portion of it, either verbatim or with modifications.  Each
licensee is addressed as ``you''.

@item
You may copy and distribute verbatim copies of the Program's source
code as you receive it, in any medium, provided that you conspicuously and
appropriately publish on each copy an appropriate copyright notice and
disclaimer of warranty; keep intact all the notices that refer to this
General Public License and to the absence of any warranty; and give any
other recipients of the Program a copy of this General Public License
along with the Program.  You may charge a fee for the physical act of
transferring a copy.

@item
You may modify your copy or copies of the Program or any portion of
it, and copy and distribute such modifications under the terms of Paragraph
1 above, provided that you also do the following:

@itemize @bullet
@item
cause the modified files to carry prominent notices stating that
you changed the files and the date of any change; and

@item
cause the whole of any work that you distribute or publish, that
in whole or in part contains the Program or any part thereof, either
with or without modifications, to be licensed at no charge to all
third parties under the terms of this General Public License (except
that you may choose to grant warranty protection to some or all
third parties, at your option).

@item
If the modified program normally reads commands interactively when
run, you must cause it, when started running for such interactive use
in the simplest and most usual way, to print or display an
announcement including an appropriate copyright notice and a notice
that there is no warranty (or else, saying that you provide a
warranty) and that users may redistribute the program under these
conditions, and telling the user how to view a copy of this General
Public License.

@item
You may charge a fee for the physical act of transferring a
copy, and you may at your option offer warranty protection in
exchange for a fee.
@end itemize

Mere aggregation of another independent work with the Program (or its
derivative) on a volume of a storage or distribution medium does not bring
the other work under the scope of these terms.

@item
You may copy and distribute the Program (or a portion or derivative of
it, under Paragraph 2) in object code or executable form under the terms of
Paragraphs 1 and 2 above provided that you also do one of the following:

@itemize @bullet
@item
accompany it with the complete corresponding machine-readable
source code, which must be distributed under the terms of
Paragraphs 1 and 2 above; or,

@item
accompany it with a written offer, valid for at least three
years, to give any third party free (except for a nominal charge
for the cost of distribution) a complete machine-readable copy of the
corresponding source code, to be distributed under the terms of
Paragraphs 1 and 2 above; or,

@item
accompany it with the information you received as to where the
corresponding source code may be obtained.  (This alternative is
allowed only for noncommercial distribution and only if you
received the program in object code or executable form alone.)
@end itemize

Source code for a work means the preferred form of the work for making
modifications to it.  For an executable file, complete source code means
all the source code for all modules it contains; but, as a special
exception, it need not include source code for modules which are standard
libraries that accompany the operating system on which the executable
file runs, or for standard header files or definitions files that
accompany that operating system.

@item
You may not copy, modify, sublicense, distribute or transfer the
Program except as expressly provided under this General Public License.
Any attempt otherwise to copy, modify, sublicense, distribute or transfer
the Program is void, and will automatically terminate your rights to use
the Program under this License.  However, parties who have received
copies, or rights to use copies, from you under this General Public
License will not have their licenses terminated so long as such parties
remain in full compliance.

@item
By copying, distributing or modifying the Program (or any work based
on the Program) you indicate your acceptance of this license to do so,
and all its terms and conditions.

@item
Each time you redistribute the Program (or any work based on the
Program), the recipient automatically receives a license from the original
licensor to copy, distribute or modify the Program subject to these
terms and conditions.  You may not impose any further restrictions on the
recipients' exercise of the rights granted herein.

@item
The Free Software Foundation may publish revised and/or new versions
of the General Public License from time to time.  Such new versions will
be similar in spirit to the present version, but may differ in detail to
address new problems or concerns.

Each version is given a distinguishing version number.  If the Program
specifies a version number of the license which applies to it and ``any
later version'', you have the option of following the terms and conditions
either of that version or of any later version published by the Free
Software Foundation.  If the Program does not specify a version number of
the license, you may choose any version ever published by the Free Software
Foundation.

@item
If you wish to incorporate parts of the Program into other free
programs whose distribution conditions are different, write to the author
to ask for permission.  For software which is copyrighted by the Free
Software Foundation, write to the Free Software Foundation; we sometimes
make exceptions for this.  Our decision will be guided by the two goals
of preserving the free status of all derivatives of our free software and
of promoting the sharing and reuse of software generally.

@iftex
@heading NO WARRANTY
@end iftex
@ifinfo
@center NO WARRANTY
@end ifinfo

@item
BECAUSE THE PROGRAM IS LICENSED FREE OF CHARGE, THERE IS NO WARRANTY
FOR THE PROGRAM, TO THE EXTENT PERMITTED BY APPLICABLE LAW.  EXCEPT WHEN
OTHERWISE STATED IN WRITING THE COPYRIGHT HOLDERS AND/OR OTHER PARTIES
PROVIDE THE PROGRAM ``AS IS'' WITHOUT WARRANTY OF ANY KIND, EITHER EXPRESSED
OR IMPLIED, INCLUDING, BUT NOT LIMITED TO, THE IMPLIED WARRANTIES OF
MERCHANTABILITY AND FITNESS FOR A PARTICULAR PURPOSE.  THE ENTIRE RISK AS
TO THE QUALITY AND PERFORMANCE OF THE PROGRAM IS WITH YOU.  SHOULD THE
PROGRAM PROVE DEFECTIVE, YOU ASSUME THE COST OF ALL NECESSARY SERVICING,
REPAIR OR CORRECTION.

@item
IN NO EVENT UNLESS REQUIRED BY APPLICABLE LAW OR AGREED TO IN WRITING WILL
ANY COPYRIGHT HOLDER, OR ANY OTHER PARTY WHO MAY MODIFY AND/OR
REDISTRIBUTE THE PROGRAM AS PERMITTED ABOVE, BE LIABLE TO YOU FOR DAMAGES,
INCLUDING ANY GENERAL, SPECIAL, INCIDENTAL OR CONSEQUENTIAL DAMAGES
ARISING OUT OF THE USE OR INABILITY TO USE THE PROGRAM (INCLUDING BUT NOT
LIMITED TO LOSS OF DATA OR DATA BEING RENDERED INACCURATE OR LOSSES
SUSTAINED BY YOU OR THIRD PARTIES OR A FAILURE OF THE PROGRAM TO OPERATE
WITH ANY OTHER PROGRAMS), EVEN IF SUCH HOLDER OR OTHER PARTY HAS BEEN
ADVISED OF THE POSSIBILITY OF SUCH DAMAGES.
@end enumerate

@iftex
@heading END OF TERMS AND CONDITIONS
@end iftex
@ifinfo
@center END OF TERMS AND CONDITIONS
@end ifinfo

@page
@unnumberedsec Appendix: How to Apply These Terms to Your New Programs

  If you develop a new program, and you want it to be of the greatest
possible use to humanity, the best way to achieve this is to make it
free software which everyone can redistribute and change under these
terms.

  To do so, attach the following notices to the program.  It is safest to
attach them to the start of each source file to most effectively convey
the exclusion of warranty; and each file should have at least the
``copyright'' line and a pointer to where the full notice is found.

@smallexample
@var{one line to give the program's name and a brief idea of what it does.}
Copyright (C) 19@var{yy}  @var{name of author}

This program is free software; you can redistribute it and/or modify
it under the terms of the GNU General Public License as published by
the Free Software Foundation; either version 1, or (at your option)
any later version.

This program is distributed in the hope that it will be useful,
but WITHOUT ANY WARRANTY; without even the implied warranty of
MERCHANTABILITY or FITNESS FOR A PARTICULAR PURPOSE.  See the
GNU General Public License for more details.

You should have received a copy of the GNU General Public License
along with this program; if not, write to the Free Software
Foundation, Inc., 675 Mass Ave, Cambridge, MA 02139, USA.
@end smallexample

Also add information on how to contact you by electronic and paper mail.

If the program is interactive, make it output a short notice like this
when it starts in an interactive mode:

@smallexample
Gnomovision version 69, Copyright (C) 19@var{yy} @var{name of author}
Gnomovision comes with ABSOLUTELY NO WARRANTY; for details type `show w'.
This is free software, and you are welcome to redistribute it
under certain conditions; type `show c' for details.
@end smallexample

The hypothetical commands `show w' and `show c' should show the
appropriate parts of the General Public License.  Of course, the
commands you use may be called something other than `show w' and `show
c'; they could even be mouse-clicks or menu items---whatever suits your
program.

You should also get your employer (if you work as a programmer) or your
school, if any, to sign a ``copyright disclaimer'' for the program, if
necessary.  Here a sample; alter the names:

@example
Yoyodyne, Inc., hereby disclaims all copyright interest in the
program `Gnomovision' (a program to direct compilers to make passes
at assemblers) written by James Hacker.

@var{signature of Ty Coon}, 1 April 1989
Ty Coon, President of Vice
@end example

That's all there is to it!

@node Concepts, Examples, Copying, Top
@chapter The Concepts of Bison

This chapter introduces many of the basic concepts without which the
details of Bison will not make sense.  If you do not already know how to
use Bison or Yacc, we suggest you start by reading this chapter carefully.

@menu
* Language and Grammar::  Languages and context-free grammars,
			    as mathematical ideas.
* Grammar in Bison::      How we represent grammars for Bison's sake.
* Semantic Values::       Each token or syntactic grouping can have
                            a semantic value (the value of an integer,
                            the name of an identifier, etc.).
* Semantic Actions::      Each rule can have an action containing C code.
* Bison Parser::          What are Bison's input and output,
                            how is the output used?
* Stages::		  Stages in writing and running Bison grammars.
* Grammar Layout::        Overall structure of a Bison grammar file.
@end menu

@node Language and Grammar, Grammar in Bison, Concepts, Concepts
@section Languages and Context-Free Grammars

@cindex context-free grammar
@cindex grammar, context-free
In order for Bison to parse a language, it must be described by a
@dfn{context-free grammar}.  This means that you specify one or more
@dfn{syntactic groupings} and give rules for constructing them from their
parts.  For example, in the C language, one kind of grouping is called an
`expression'.  One rule for making an expression might be, ``An expression
can be made of a minus sign and another expression''.  Another would be,
``An expression can be an integer''.  As you can see, rules are often
recursive, but there must be at least one rule which leads out of the
recursion.

@cindex BNF
@cindex Backus-Naur form
The most common formal system for presenting such rules for humans to read
is @dfn{Backus-Naur Form} or ``BNF'', which was developed in order to
specify the language Algol 60.  Any grammar expressed in BNF is a
context-free grammar.  The input to Bison is essentially machine-readable
BNF.

@cindex symbols (abstract)
@cindex token
@cindex syntactic grouping
@cindex grouping, syntactic
In the formal grammatical rules for a language, each kind of syntactic unit
or grouping is named by a @dfn{symbol}.  Those which are built by grouping
smaller constructs according to grammatical rules are called
@dfn{nonterminal symbols}; those which can't be subdivided are called
@dfn{terminal symbols} or @dfn{token types}.  We call a piece of input
corresponding to a single terminal symbol a @dfn{token}, and a piece
corresponding to a single nonterminal symbol a @dfn{grouping}.@refill

We can use the C language as an example of what symbols, terminal and
nonterminal, mean.  The tokens of C are identifiers, constants (numeric and
string), and the various keywords, arithmetic operators and punctuation
marks.  So the terminal symbols of a grammar for C include `identifier',
`number', `string', plus one symbol for each keyword, operator or
punctuation mark: `if', `return', `const', `static', `int', `char',
`plus-sign', `open-brace', `close-brace', `comma' and many more.  (These
tokens can be subdivided into characters, but that is a matter of
lexicography, not grammar.)

Here is a simple C function subdivided into tokens:

@example
int             /* @r{keyword `int'} */
square (x)      /* @r{identifier, open-paren,} */
                /* @r{identifier, close-paren} */
     int x;     /* @r{keyword `int', identifier, semicolon} */
@{               /* @r{open-brace} */
  return x * x; /* @r{keyword `return', identifier,} */
                /* @r{asterisk, identifier, semicolon} */
@}               /* @r{close-brace} */
@end example

The syntactic groupings of C include the expression, the statement, the
declaration, and the function definition.  These are represented in the
grammar of C by nonterminal symbols `expression', `statement',
`declaration' and `function definition'.  The full grammar uses dozens of
additional language constructs, each with its own nonterminal symbol, in
order to express the meanings of these four.  The example above is a
function definition; it contains one declaration, and one statement.  In
the statement, each @samp{x} is an expression and so is @samp{x * x}.

Each nonterminal symbol must have grammatical rules showing how it is made
out of simpler constructs.  For example, one kind of C statement is the
@code{return} statement; this would be described with a grammar rule which
reads informally as follows:

@quotation
A `statement' can be made of a `return' keyword, an `expression' and a
`semicolon'.
@end quotation

@noindent
There would be many other rules for `statement', one for each kind of
statement in C.

@cindex start symbol
One nonterminal symbol must be distinguished as the special one which
defines a complete utterance in the language.  It is called the @dfn{start
symbol}.  In a compiler, this means a complete input program.  In the C
language, the nonterminal symbol `sequence of definitions and declarations'
plays this role.

For example, @samp{1 + 2} is a valid C expression---a valid part of a C
program---but it is not valid as an @emph{entire} C program.  In the
context-free grammar of C, this follows from the fact that `expression' is
not the start symbol.

The Bison parser reads a sequence of tokens as its input, and groups the
tokens using the grammar rules.  If the input is valid, the end result is
that the entire token sequence reduces to a single grouping whose symbol is
the grammar's start symbol.  If we use a grammar for C, the entire input
must be a `sequence of definitions and declarations'.  If not, the parser
reports a syntax error.

@node Grammar in Bison, Semantic Values, Language and Grammar, Concepts
@section From Formal Rules to Bison Input
@cindex Bison grammar
@cindex formal grammar

A formal grammar is a mathematical construct.  To define the language
for Bison, you must write a file expressing the grammar in Bison syntax:
a @dfn{Bison grammar} file.  @xref{Grammar File}.

A nonterminal symbol in the formal grammar is represented in Bison input
as an identifier, like an identifier in C.  By convention, it should be
in lower case, such as @code{expr}, @code{stmt} or @code{declaration}.

The Bison representation for a terminal symbol is also called a @dfn{token
type}.  Token types as well can be represented as C-like identifiers.  By
convention, these identifiers should be upper case to distinguish them from
nonterminals: for example, @code{INTEGER}, @code{IDENTIFIER}, @code{IF} or
@code{RETURN}.  A terminal symbol that stands for a particular keyword in
the language should be named after that keyword converted to upper case.
The terminal symbol @code{error} is reserved for error recovery.
@xref{Symbols}.@refill

A terminal symbol can also be represented as a character literal, just like
a C character constant.  You should do this whenever a token is just a
single character (parenthesis, plus-sign, etc.): use that same character in
a literal as the terminal symbol for that token.

The grammar rules also have an expression in Bison syntax.  For example,
here is the Bison rule for a C @code{return} statement.  The semicolon in
quotes is a literal character token, representing part of the C syntax for
the statement; the naked semicolon, and the colon, are Bison punctuation
used in every rule.

@example
stmt:   RETURN expr ';'
        ;
@end example

@noindent
@xref{Rules}.

@node Semantic Values, Semantic Actions, Grammar in Bison, Concepts
@section Semantic Values
@cindex semantic value

A formal grammar selects tokens only by their classifications: for example,
if a rule mentions the terminal symbol `integer constant', it means that
@emph{any} integer constant is grammatically valid in that position.  The
precise value of the constant is irrelevant to how to parse the input: if
@samp{x+4} is grammatical then @samp{x+1} or @samp{x+3989} is equally
grammatical.@refill

But the precise value is very important for what the input means once it is
parsed.  A compiler is useless if it fails to distinguish between 4, 1 and
3989 as constants in the program!  Therefore, each token in a Bison grammar
has both a token type and a @dfn{semantic value}.  @xref{Semantics},
for details.

The token type is a terminal symbol defined in the grammar, such as
@code{INTEGER_CONSTANT}, @code{IDENTIFIER} or @code{','}.  It tells
everything you need to know to decide where the token may validly appear
and how to group it with other tokens.  The grammar rules know nothing
about tokens except their types.@refill

The semantic value has all the the rest of the information about the
meaning of the token, such as the value of an integer, or the name of an
identifier.  (A token such as @code{','} which is just punctuation doesn't
need to have any semantic value.)

For example, an input token might be classified as token type
@code{INTEGER} and have the semantic value 4.  Another input token might
have the same token type @code{INTEGER} but value 3989.  When a grammar
rule says that @code{INTEGER} is allowed, either of these tokens is
acceptable because each is an @code{INTEGER}.  When the parser accepts the
token, it keeps track of the token's semantic value.

Each grouping can also have a semantic value as well as its nonterminal
symbol.  For example, in a calculator, an expression typically has a
semantic value that is a number.  In a compiler for a programming
language, an expression typically has a semantic value that is a tree
structure describing the meaning of the expression.

@node Semantic Actions, Bison Parser, Semantic Values, Concepts
@section Semantic Actions
@cindex semantic actions
@cindex actions, semantic

In order to be useful, a program must do more than parse input; it must
also produce some output based on the input.  In a Bison grammar, a grammar
rule can have an @dfn{action} made up of C statements.  Each time the
parser recognizes a match for that rule, the action is executed.
@xref{Actions}.
	
Most of the time, the purpose of an action is to compute the semantic value
of the whole construct from the semantic values of its parts.  For example,
suppose we have a rule which says an expression can be the sum of two
expressions.  When the parser recognizes such a sum, each of the
subexpressions has a semantic value which describes how it was built up.
The action for this rule should create a similar sort of value for the
newly recognized larger expression.

For example, here is a rule that says an expression can be the sum of
two subexpressions:

@example
expr: expr '+' expr   @{ $$ = $1 + $3; @}
        ;
@end example

@noindent
The action says how to produce the semantic value of the sum expression
from the values of the two subexpressions.

@node Bison Parser, Stages, Semantic Actions, Concepts
@section Bison Output: the Parser File
@cindex Bison parser
@cindex Bison utility
@cindex lexical analyzer, purpose
@cindex parser

When you run Bison, you give it a Bison grammar file as input.  The output
is a C source file that parses the language described by the grammar.
This file is called a @dfn{Bison parser}.  Keep in mind that the Bison
utility and the Bison parser are two distinct programs: the Bison utility
is a program whose output is the Bison parser that becomes part of your
program.

The job of the Bison parser is to group tokens into groupings according to
the grammar rules---for example, to build identifiers and operators into
expressions.  As it does this, it runs the actions for the grammar rules it
uses.

The tokens come from a function called the @dfn{lexical analyzer} that you
must supply in some fashion (such as by writing it in C).  The Bison parser
calls the lexical analyzer each time it wants a new token.  It doesn't know
what is ``inside'' the tokens (though their semantic values may reflect
this).  Typically the lexical analyzer makes the tokens by parsing
characters of text, but Bison does not depend on this.  @xref{Lexical}.

The Bison parser file is C code which defines a function named
@code{yyparse} which implements that grammar.  This function does not make
a complete C program: you must supply some additional functions.  One is
the lexical analyzer.  Another is an error-reporting function which the
parser calls to report an error.  In addition, a complete C program must
start with a function called @code{main}; you have to provide this, and
arrange for it to call @code{yyparse} or the parser will never run.
@xref{Interface}.

Aside from the token type names and the symbols in the actions you
write, all variable and function names used in the Bison parser file
begin with @samp{yy} or @samp{YY}.  This includes interface functions
such as the lexical analyzer function @code{yylex}, the error reporting
function @code{yyerror} and the parser function @code{yyparse} itself.
This also includes numerous identifiers used for internal purposes.
Therefore, you should avoid using C identifiers starting with @samp{yy}
or @samp{YY} in the Bison grammar file except for the ones defined in
this manual.

@node Stages, Grammar Layout, Bison Parser, Concepts
@section Stages in Using Bison
@cindex stages in using Bison

The actual language-design process using Bison, from grammar specification
to a working compiler or interpreter, has these parts:

@enumerate
@item
Formally specify the grammar in a form recognized by Bison
(@pxref{Grammar File}).  For each grammatical rule in the language,
describe the action that is to be taken when an instance of that rule
is recognized.  The action is described by a sequence of C statements.

@item
Write a lexical analyzer to process input and pass tokens to the
parser.  The lexical analyzer may be written by hand in C
(@pxref{Lexical}).  It could also be produced using Lex, but the use
of Lex is not discussed in this manual.

@item
Write a controlling function that calls the Bison-produced parser.

@item
Write error-reporting routines.
@end enumerate

To turn this source code as written into a runnable program, you
must follow these steps:

@enumerate
@item
Run Bison on the grammar to produce the parser.

@item
Compile the code output by Bison, as well as any other source files.

@item
Link the object files to produce the finished product.
@end enumerate

@node Grammar Layout,, Stages, Concepts
@section The Overall Layout of a Bison Grammar
@cindex grammar file
@cindex layout of Bison grammar

The input file for the Bison utility is a @dfn{Bison grammar file}.  The
general form of a Bison grammar file is as follows:

@example
%@{
@var{C declarations}
%@}

@var{Bison declarations}

%%
@var{Grammar rules}
%%
@var{Additional C code}
@end example

@noindent
The @samp, @samp{%@{} and @samp{%@}} are punctuation that appears
in every Bison grammar file to separate the sections.

The C declarations may define types and variables used in the actions.
You can also use preprocessor commands to define macros used there, and use
@code{#include} to include header files that do any of these things.

The Bison declarations declare the names of the terminal and nonterminal
symbols, and may also describe operator precedence and the data types of
semantic values of various symbols.

The grammar rules define how to construct each nonterminal symbol from its
parts.

The additional C code can contain any C code you want to use.  Often the
definition of the lexical analyzer @code{yylex} goes here, plus subroutines
called by the actions in the grammar rules.  In a simple program, all the
rest of the program can go here.

@node Examples, Grammar File, Concepts, Top
@chapter Examples
@cindex simple examples
@cindex examples, simple

Now we show and explain three sample programs written using Bison: a
reverse polish notation calculator, an algebraic (infix) notation
calculator, and a multi-function calculator.  All three have been tested
under BSD Unix 4.3; each produces a usable, though limited, interactive
desk-top calculator.

These examples are simple, but Bison grammars for real programming
languages are written the same way.
@ifinfo
You can copy these examples out of the Info file and into a source file
to try them.
@end ifinfo

@menu
* RPN Calc::               Reverse polish notation calculator;
			     a first example with no operator precedence.
* Infix Calc::		   Infix (algebraic) notation calculator.
			     Operator precedence is introduced.
* Simple Error Recovery::  Continuing after syntax errors.
* Multi-function Calc::    Calculator with memory and trig functions.
			     It uses multiple data-types for semantic values.
* Exercises::		   Ideas for improving the multi-function calculator.
@end menu

@node RPN Calc, Infix Calc, Examples, Examples
@section Reverse Polish Notation Calculator
@cindex reverse polish notation
@cindex polish notation calculator
@cindex @code{rpcalc}
@cindex calculator, simple

The first example is that of a simple double-precision @dfn{reverse polish
notation} calculator (a calculator using postfix operators).  This example
provides a good starting point, since operator precedence is not an issue.
The second example will illustrate how operator precedence is handled.

The source code for this calculator is named @file{rpcalc.y}.  The
@samp{.y} extension is a convention used for Bison input files.

@menu
* Decls: Rpcalc Decls.    Bison and C declarations for rpcalc.
* Rules: Rpcalc Rules.    Grammar Rules for rpcalc, with explanation.
* Input: Rpcalc Input.	  Explaining the rules for @code{input}.
* Line: Rpcalc Line.	  Explaining the rules for @code{line}.
* Expr: Rpcalc Expr.      Explaining the rules for @code{expr}.
* Lexer: Rpcalc Lexer.    The lexical analyzer.
* Main: Rpcalc Main.      The controlling function.
* Error: Rpcalc Error.    The error reporting function.
* Gen: Rpcalc Gen.        Running Bison on the grammar file.
* Comp: Rpcalc Compile.   Run the C compiler on the output code.
@end menu

@node Rpcalc Decls, Rpcalc Rules, RPN calc, RPN calc
@subsection Declarations for Rpcalc

Here are the C and Bison declarations for the reverse polish notation
calculator.  As in C, comments are placed between @samp{/*@dots{}*/}.

@example
/* Reverse polish notation calculator. */

%@{
#define YYSTYPE double
#include <math.h>
%@}

%token NUM

%% /* Grammar rules and actions follow */
@end example

The C declarations section (@pxref{C Declarations}) contains two
preprocessor directives.

The @code{#define} directive defines the macro @code{YYSTYPE}, thus
specifying the C data type for semantic values of both tokens and groupings
(@pxref{Value Type}).  The Bison parser will use whatever type
@code{YYSTYPE} is defined as; if you don't define it, @code{int} is the
default.  Because we specify @code{double}, each token and each expression
has an associated value, which is a floating point number.

The @code{#include} directive is used to declare the exponentiation
function @code{pow}.

The second section, Bison declarations, provides information to Bison about
the token types (@pxref{Bison Declarations}).  Each terminal symbol that is
not a single-character literal must be declared here.  (Single-character
literals normally don't need to be declared.)  In this example, all the
arithmetic operators are designated by single-character literals, so the
only terminal symbol that needs to be declared is @code{NUM}, the token
type for numeric constants.

@node Rpcalc Rules, Rpcalc Input, Rpcalc Decls, RPN Calc
@subsection Grammar Rules for Rpcalc

Here are the grammar rules for the reverse polish notation calculator.

@example
input:    /* empty */
        | input line
;

line:     '\n'
        | exp '\n'  @{ printf ("\t%.10g\n", $1); @}
;

exp:      NUM             @{ $$ = $1;         @}
        | exp exp '+'     @{ $$ = $1 + $2;    @}
        | exp exp '-'     @{ $$ = $1 - $2;    @}
        | exp exp '*'     @{ $$ = $1 * $2;    @}
        | exp exp '/'     @{ $$ = $1 / $2;    @}
      /* Exponentiation */
        | exp exp '^'     @{ $$ = pow ($1, $2); @}
      /* Unary minus    */
        | exp 'n'         @{ $$ = -$1;        @}
;
%%
@end example

The groupings of the rpcalc ``language'' defined here are the expression
(given the name @code{exp}), the line of input (@code{line}), and the
complete input transcript (@code{input}).  Each of these nonterminal
symbols has several alternate rules, joined by the @samp{|} punctuator
which is read as ``or''.  The following sections explain what these rules
mean.

The semantics of the language is determined by the actions taken when a
grouping is recognized.  The actions are the C code that appears inside
braces.  @xref{Actions}.

You must specify these actions in C, but Bison provides the means for
passing semantic values between the rules.  In each action, the
pseudo-variable @code{$$} stands for the semantic value for the grouping
that the rule is going to construct.  Assigning a value to @code{$$} is the
main job of most actions.  The semantic values of the components of the
rule are referred to as @code{$1}, @code{$2}, and so on.

@node Rpcalc Input, Rpcalc Line, Rpcalc Rules, RPN Calc
@subsubsection Explanation of @code{input}

Consider the definition of @code{input}:

@example
input:    /* empty */
        | input line
;
@end example

This definition reads as follows: ``A complete input is either an empty
string, or a complete input followed by an input line''.  Notice that
``complete input'' is defined in terms of itself.  This definition is said
to be @dfn{left recursive} since @code{input} appears always as the
leftmost symbol in the sequence.  @xref{Recursion}.

The first alternative is empty because there are no symbols between the
colon and the first @samp{|}; this means that @code{input} can match an
empty string of input (no tokens).  We write the rules this way because it
is legitimate to type @kbd{Ctrl-d} right after you start the calculator.
It's conventional to put an empty alternative first and write the comment
@samp{/* empty */} in it.

The second alternate rule (@code{input line}) handles all nontrivial input.
It means, ``After reading any number of lines, read one more line if
possible.''  The left recursion makes this rule into a loop.  Since the
first alternative matches empty input, the loop can be executed zero or
more times.

The parser function @code{yyparse} continues to process input until a
grammatical error is seen or the lexical analyzer says there are no more
input tokens; we will arrange for the latter to happen at end of file.

@node Rpcalc Line, Rpcalc Expr, Rpcalc Input, RPN Calc
@subsubsection Explanation of @code{line}

Now consider the definition of @code{line}:

@example
line:     '\n'
        | exp '\n'  @{ printf ("\t%.10g\n", $1); @}
;
@end example

The first alternative is a token which is a newline character; this means
that rpcalc accepts a blank line (and ignores it, since there is no
action).  The second alternative is an expression followed by a newline.
This is the alternative that makes rpcalc useful.  The semantic value of
the @code{exp} grouping is the value of @code{$1} because the @code{exp} in
question is the first symbol in the alternative.  The action prints this
value, which is the result of the computation the user asked for.

This action is unusual because it does not assign a value to @code{$$}.  As
a consequence, the semantic value associated with the @code{line} is
uninitialized (its value will be unpredictable).  This would be a bug if
that value were ever used, but we don't use it: once rpcalc has printed the
value of the user's input line, that value is no longer needed.

@node Rpcalc Expr, Rpcalc Lexer, Rpcalc Line, RPN Calc
@subsubsection Explanation of @code{expr}

The @code{exp} grouping has several rules, one for each kind of expression.
The first rule handles the simplest expressions: those that are just numbers.
The second handles an addition-expression, which looks like two expressions
followed by a plus-sign.  The third handles subtraction, and so on.

@example
exp:      NUM
        | exp exp '+'     @{ $$ = $1 + $2;    @}
        | exp exp '-'     @{ $$ = $1 - $2;    @}
        @dots{}
        ;
@end example

We have used @samp{|} to join all the rules for @code{exp}, but we could
equally well have written them separately:

@example
exp:      NUM ;
exp:      exp exp '+'     @{ $$ = $1 + $2;    @} ;
exp:      exp exp '-'     @{ $$ = $1 - $2;    @} ;
        @dots{}
@end example

Most of the rules have actions that compute the value of the expression in
terms of the value of its parts.  For example, in the rule for addition,
@code{$1} refers to the first component @code{exp} and @code{$2} refers to
the second one.  The third component, @code{'+'}, has no meaningful
associated semantic value, but if it had one you could refer to it as
@code{$3}.  When @code{yyparse} recognizes a sum expression using this
rule, the sum of the two subexpressions' values is produced as the value of
the entire expression.  @xref{Actions}.

You don't have to give an action for every rule.  When a rule has no
action, Bison by default copies the value of @code{$1} into @code{$$}.
This is what happens in the first rule (the one that uses @code{NUM}).

The formatting shown here is the recommended convention, but Bison does
not require it.  You can add or change whitespace as much as you wish.
For example, this:

@example
exp   : NUM | exp exp '+' @{$$ = $1 + $2; @} | @dots{}
@end example

@noindent
means the same thing as this:

@example
exp:      NUM
        | exp exp '+'    @{ $$ = $1 + $2; @}
        | @dots{}
@end example

@noindent
The latter, however, is much more readable.

@node Rpcalc Lexer, Rpcalc Main, Rpcalc Expr, RPN Calc
@subsection The Rpcalc Lexical Analyzer
@cindex writing a lexical analyzer
@cindex lexical analyzer, writing

The lexical analyzer's job is low-level parsing: converting characters or
sequences of characters into tokens.  The Bison parser gets its tokens by
calling the lexical analyzer.  @xref{Lexical}.

Only a simple lexical analyzer is needed for the RPN calculator.  This
lexical analyzer skips blanks and tabs, then reads in numbers as
@code{double} and returns them as @code{NUM} tokens.  Any other character
that isn't part of a number is a separate token.  Note that the token-code
for such a single-character token is the character itself.

The return value of the lexical analyzer function is a numeric code which
represents a token type.  The same text used in Bison rules to stand for
this token type is also a C expression for the numeric code for the type.
This works in two ways.  If the token type is a character literal, then its
numeric code is the ASCII code for that character; you can use the same
character literal in the lexical analyzer to express the number.  If the
token type is an identifier, that identifier is defined by Bison as a C
macro whose definition is the appropriate number.  In this example,
therefore, @code{NUM} becomes a macro for @code{yylex} to use.

The semantic value of the token (if it has one) is stored into the global
variable @code{yylval}, which is where the Bison parser will look for it.
(The C data type of @code{yylval} is @code{YYSTYPE}, which was defined
at the beginning of the grammar; @pxref{Rpcalc Decls}.)

A token type code of zero is returned if the end-of-file is encountered.
(Bison recognizes any nonpositive value as indicating the end of the
input.)

Here is the code for the lexical analyzer:

@example
/* Lexical analyzer returns a double floating point number on the
   stack and the token NUM, or the ASCII character read if not a
   number.  Skips all blanks and tabs, returns 0 for EOF. */

#include <ctype.h>

yylex ()
@{
  int c;

  while ((c = getchar ()) == ' ' || c == '\t')  /* skip white space  */
    ;
  if (c == '.' || isdigit (c))                /* process numbers   */
    @{
      ungetc (c, stdin);
      scanf ("%lf", &yylval);
      return NUM;
    @}
  if (c == EOF)                            /* return end-of-file  */
    return 0;
  return c;                                /* return single chars */
@}
@end example

@node Rpcalc Main, Rpcalc Error, Rpcalc Lexer, RPN Calc
@subsection The Controlling Function
@cindex controlling function
@cindex main function in simple example

In keeping with the spirit of this example, the controlling function is
kept to the bare minimum.  The only requirement is that it call
@code{yyparse} to start the process of parsing.

@example
main ()
@{
  yyparse ();
@}
@end example

@node Rpcalc Error, Rpcalc Gen, Rpcalc Main, RPN Calc
@subsection The Error Reporting Routine
@cindex error reporting routine
@findex yyerror

When @code{yyparse} detects a syntax error, it calls the error reporting
function @code{yyerror} to print an error message (usually but not always
@code{"parse error"}).  It is up to the programmer to supply @code{yyerror}
(@pxref{Interface}), so here is the definition we will use:

@example
#include <stdio.h>

yyerror (s)  /* Called by yyparse on error */
     char *s;
@{
  printf ("%s\n", s);
@}
@end example

After @code{yyerror} returns, the Bison parser may recover from the error
and continue parsing if the grammar contains a suitable error rule
(@pxref{Error Recovery}).  Otherwise, @code{yyparse} returns nonzero.  We
have not written any error rules in this example, so any invalid input will
cause the calculator program to exit.  This is not clean behavior for a
real calculator, but it is adequate in the first example.

@node Rpcalc Gen, Rpcalc Compile, Rpcalc Error, RPN Calc
@subsection Running Bison to Make the Parser
@cindex running Bison (introduction)

Before running Bison to produce a parser, we need to decide how to arrange
all the source code in one or more source files.  For such a simple example,
the easiest thing is to put everything in one file.  The definitions of
@code{yylex}, @code{yyerror} and @code{main} go at the end, in the
``additional C code'' section of the file (@pxref{Grammar Layout}).

For a large project, you would probably have several source files, and use
@code{make} to arrange to recompile them.

With all the source in a single file, you use the following command to
convert it into a parser file:

@example
bison @var{file_name}.y
@end example

@noindent
In this example the file was called @file{rpcalc.y} (for ``Reverse Polish
CALCulator'').  Bison produces a file named @file{@var{file_name}.tab.c},
removing the @samp{.y} from the original file name. The file output by
Bison contains the source code for @code{yyparse}.  The additional
functions in the input file (@code{yylex}, @code{yyerror} and @code{main})
are copied verbatim to the output.

@node Rpcalc Compile,, Rpcalc Gen, RPN Calc
@subsection Compiling the Parser File
@cindex compiling the parser

Here is how to compile and run the parser file:

@example
# @r{List files in current directory.}
% ls
rpcalc.tab.c  rpcalc.y

# @r{Compile the Bison parser.}
# @r{@samp{-lm} tells compiler to search math library for @code{pow}.}
% cc rpcalc.tab.c -lm -o rpcalc

# @r{List files again.}
% ls
rpcalc  rpcalc.tab.c  rpcalc.y
@end example

The file @file{rpcalc} now contains the executable code.  Here is an
example session using @code{rpcalc}.

@example
% rpcalc
4 9 +
13
3 7 + 3 4 5 *+-
-13
3 7 + 3 4 5 * + - n              @r{Note the unary minus, @samp{n}}
13
5 6 / 4 n +
-3.166666667
3 4 ^                            @r{Exponentiation}
81
^D                               @r{End-of-file indicator}
%
@end example

@node Infix Calc, Simple Error Recovery, RPN Calc, Top
@section Infix Notation Calculator: @code{calc}
@cindex infix notation calculator
@cindex @code{calc}
@cindex calculator, infix notation

We now modify rpcalc to handle infix operators instead of postfix.  Infix
notation involves the concept of operator precedence and the need for
parentheses nested to arbitrary depth.  Here is the Bison code for
@file{calc.y}, an infix desk-top calculator.

@example
/* Infix notation calculator--calc */

%@{
#define YYSTYPE double
#include <math.h>
%@}

%token NUM
%left '-' '+'
%left '*' '/'
%left NEG     /* negation--unary minus */
%right '^'    /* exponentiation        */

/* Grammar follows */
%%
input:    /* empty string */
        | input line
;

line:     '\n'
        | exp '\n'  @{ printf("\t%.10g\n", $1); @}
;

exp:      NUM                @{ $$ = $1;         @}
        | exp '+' exp        @{ $$ = $1 + $3;    @}
        | exp '-' exp        @{ $$ = $1 - $3;    @}
        | exp '*' exp        @{ $$ = $1 * $3;    @}
        | exp '/' exp        @{ $$ = $1 / $3;    @}
        | '-' exp  %prec NEG @{ $$ = -$2;        @}
        | exp '^' exp        @{ $$ = pow ($1, $3); @}
        | '(' exp ')'        @{ $$ = $2;         @}
;
%%
@end example

@noindent
The functions @code{yylex}, @code{yyerror} and @code{main} can be the same
as before.

There are two important new features shown in this code.

In the second section (Bison declarations), @code{%left} declares token
types and says they are left-associative operators.  The declarations
@code{%left} and @code{%right} (right associativity) take the place of
@code{%token} which is used to declare a token type name without
associativity.  (These tokens are single-character literals, which
ordinarily don't need to be declared.  We declare them here to specify
the associativity.)

Operator precedence is determined by the line ordering of the declarations;
the lower the declaration, the higher the precedence.  Hence,
exponentiation has the highest precedence, unary minus (@code{NEG}) is next,
followed by @samp{*} and @samp{/}, and so on.  @xref{Precedence}.

The other important new feature is the @code{%prec} in the grammar section
for the unary minus operator.  The @code{%prec} simply instructs Bison that
the rule @samp{| '-' exp} has the same precedence as @code{NEG}---in this
case the next-to-highest.  @xref{Contextual Precedence}.

Here is a sample run of @file{calc.y}:

@example
% calc
4 + 4.5 - (34/(8*3+-3))
6.880952381
-56 + 2
-54
3 ^ 2
9
@end example

@node Simple Error Recovery, Multi-function Calc, Infix Calc, Examples
@section Simple Error Recovery
@cindex error recovery, simple

Up to this point, this manual has not addressed the issue of @dfn{error
recovery}---how to continue parsing after the parser detects a syntax
error.  All we have handled is error reporting with @code{yyerror}.  Recall
that by default @code{yyparse} returns after calling @code{yyerror}.  This
means that an erroneous input line causes the calculator program to exit.
Now we show how to rectify this deficiency.

The Bison language itself includes the reserved word @code{error}, which
may be included in the grammar rules.  In the example below it has
been added to one of the alternatives for @code{line}:

@example
line:     '\n'
        | exp '\n'   @{ printf("\t%.10g\n", $1); @}
        | error '\n' @{ yyerrok;                 @}
;
@end example

This addition to the grammar allows for simple error recovery in the event
of a parse error.  If an expression that cannot be evaluated is read, the
error will be recognized by the third rule for @code{line}, and parsing
will continue.  (The @code{yyerror} function is still called upon to print
its message as well.)  The action executes the statement @code{yyerrok}, a
macro defined automatically by Bison; its meaning is that error recovery is
complete (@pxref{Error Recovery}).  Note the difference between
@code{yyerrok} and @code{yyerror}; neither one is a misprint.@refill

This form of error recovery deals with syntax errors.  There are other
kinds of errors; for example, division by zero, which raises an exception
signal that is normally fatal.  A real calculator program must handle this
signal and use @code{longjmp} to return to @code{main} and resume parsing
input lines; it would also have to discard the rest of the current line of
input.  We won't discuss this issue further because it is not specific to
Bison programs.

@node Multi-function Calc,, Simple Error Recovery, Examples
@section Multi-Function Calculator: @code{mfcalc}
@cindex multi-function calculator
@cindex @code{mfcalc}
@cindex calculator, multi-function

Now that the basics of Bison have been discussed, it is time to move on to
a more advanced problem.  The above calculators provided only five
functions, @samp{+}, @samp{-}, @samp{*}, @samp{/} and @samp{^}.  It would
be nice to have a calculator that provides other mathematical functions such
as @code{sin}, @code{cos}, etc.

It is easy to add new operators to the infix calculator as long as they are
only single-character literals.  The lexical analyzer @code{yylex} passes
back all non-number characters as tokens, so new grammar rules suffice for
adding a new operator.  But we want something more flexible: built-in
functions whose syntax has this form:

@example
@var{function_name} (@var{argument})
@end example

@noindent
At the same time, we will add memory to the calculator, by allowing you
to create named variables, store values in them, and use them later.
Here is a sample session with the multi-function calculator:

@example
% acalc
pi = 3.141592653589
3.1415926536
sin(pi)
0.0000000000
alpha = beta1 = 2.3
2.3000000000
alpha
2.3000000000
ln(alpha)
0.8329091229
exp(ln(beta1))
2.3000000000
%
@end example

Note that multiple assignment and nested function calls are permitted.

@menu
* Decl: Mfcalc Decl.     Bison declarations for multi-function calculator.
* Rules: Mfcalc Rules.   Grammar rules for the calculator.
* Symtab: Mfcalc Symtab. Symbol table management subroutines.
@end menu

@node Mfcalc Decl, Mfcalc Rules, Multi-function Calc, Multi-function Calc
@subsection Declarations for @code{mfcalc}

Here are the C and Bison declarations for the multi-function calculator.

@example
%@{
#include <math.h>  /* For math functions, cos(), sin(), etc      */
#include "calc.h"  /* Contains definition of `symrec' */
%@}
%union @{
double     val;  /* For returning numbers.              */
symrec  *tptr;   /* For returning symbol-table pointers */
@}

%token <val>  NUM        /* Simple double precision number  */
%token <tptr> VAR FNCT   /* Variable and Function           */
%type  <val>  exp

%right '='
%left '-' '+'
%left '*' '/'
%left NEG     /* Negation--unary minus */
%right '^'    /* Exponentiation        */

/* Grammar follows */

%%
@end example

The above grammar introduces only two new features of the Bison language.
These features allow semantic values to have various data types
(@pxref{Multiple Types}).

The @code{%union} declaration specifies the entire list of possible types;
this is instead of defining @code{YYSTYPE}.  The allowable types are now
double-floats (for @code{exp} and @code{NUM}) and pointers to entries in
the symbol table.  @xref{Union Decl}.

Since values can now have various types, it is necessary to associate a
type with each grammar symbol whose semantic value is used.  These symbols
are @code{NUM}, @code{VAR}, @code{FNCT}, and @code{exp}.  Their
declarations are augmented with information about their data type (placed
between angle brackets).

The Bison construct @code{%type} is used for declaring nonterminal symbols,
just as @code{%token} is used for declaring token types.  We have not used
@code{%type} before because nonterminal symbols are normally declared
implicitly by the rules that define them.  But @code{exp} must be declared
explicitly so we can specify its value type.  @xref{Type Decl}.

@node Mfcalc Rules, Mfcalc Symtab, Mfcalc Decl, Multi-function Calc
@subsection Grammar Rules for @code{mfcalc}

Here are the grammar rules for the multi-function calculator.
Most of them are copied directly from @code{calc}; three rules,
those which mention @code{VAR} or @code{FNCT}, are new.

@example
input:   /* empty */
        | input line
;

line:
          '\n'
        | exp '\n'   @{ printf ("\t%.10g\n", $1); @}
        | error '\n' @{ yyerrok;                  @}
;

exp:      NUM                @{ $$ = $1;                         @}
        | VAR                @{ $$ = $1->value.var;              @}
        | VAR '=' exp        @{ $$ = $3; $1->value.var = $3;     @}
        | FNCT '(' exp ')'   @{ $$ = (*($1->value.fnctptr))($3); @}
        | exp '+' exp        @{ $$ = $1 + $3;                    @}
        | exp '-' exp        @{ $$ = $1 - $3;                    @}
        | exp '*' exp        @{ $$ = $1 * $3;                    @}
        | exp '/' exp        @{ $$ = $1 / $3;                    @}
        | '-' exp  %prec NEG @{ $$ = -$2;                        @}
        | exp '^' exp        @{ $$ = pow ($1, $3);               @}
        | '(' exp ')'        @{ $$ = $2;                         @}
;
/* End of grammar */
%%
@end example

@node Mfcalc Symtab,, Mfcalc Rules, Multi-function Calc
@subsection Managing the Symbol Table for @code{mfcalc}
@cindex symbol table example

The multi-function calculator requires a symbol table to keep track of the
names and meanings of variables and functions.  This doesn't affect the
grammar rules (except for the actions) or the Bison declarations, but it
requires some additional C functions for support.

The symbol table itself consists of a linked list of records.  Its
definition, which is kept in the header @file{calc.h}, is as follows.  It
provides for either functions or variables to be placed in the table.

@example
/* Data type for links in the chain of symbols.  */
struct symrec
@{
  char *name;  /* name of symbol              */
  int type;    /* type of symbol: either VAR or FNCT */
  union @{
    double var;           /* value of a VAR  */
    double (*fnctptr)();  /* value of a FNCT */
  @} value;
  struct symrec *next;    /* link field    */
@};

typedef struct symrec symrec;

/* The symbol table: a chain of `struct symrec'.  */
extern symrec *sym_table;

symrec *putsym ();
symrec *getsym ();
@end example

The new version of @code{main} includes a call to @code{init_table}, a
function that initializes the symbol table.  Here it is, and
@code{init_table} as well:

@example
#include <stdio.h>

main()
@{
  init_table ();
  yyparse ();
@}

yyerror (s)  /* Called by yyparse on error */
     char *s;
@{
  printf ("%s\n", s);
@}

struct init
@{
  char *fname;
  double (*fnct)();
@};

struct init arith_fncts[]
  = @{
      "sin", sin,
      "cos", cos,
      "atan", atan,
      "ln", log,
      "exp", exp,
      "sqrt", sqrt,
      0, 0
    @};

/* The symbol table: a chain of `struct symrec'.  */
symrec *sym_table = (symrec *)0;

init_table ()  /* puts arithmetic functions in table. */
@{
  int i;
  symrec *ptr;
  for (i = 0; arith_fncts[i].fname != 0; i++)
    @{
      ptr = putsym (arith_fncts[i].fname, FNCT);
      ptr->value.fnctptr = arith_fncts[i].fnct;
    @}
@}
@end example

By simply editing the initialization list and adding the necessary include
files, you can add additional functions to the calculator.

Two important functions allow look-up and installation of symbols in the
symbol table.  The function @code{putsym} is passed a name and the type
(@code{VAR} or @code{FNCT}) of the object to be installed.  The object is
linked to the front of the list, and a pointer to the object is returned.
The function @code{getsym} is passed the name of the symbol to look up.  If
found, a pointer to that symbol is returned; otherwise zero is returned.

@example
symrec *
putsym (sym_name,sym_type)
     char *sym_name;
     int sym_type;
@{
  symrec *ptr;
  ptr = (symrec *) malloc (sizeof(symrec));
  ptr->name = (char *) malloc (strlen(sym_name)+1);
  strcpy (ptr->name,sym_name);
  ptr->type = sym_type;
  ptr->value.var = 0; /* set value to 0 even if fctn.  */
  ptr->next = (struct symrec *)sym_table;
  sym_table = ptr;
  return ptr;
@}

symrec *
getsym (sym_name)
     char *sym_name;
@{
  symrec *ptr;
  for (ptr = sym_table; ptr != (symrec *) 0;
       ptr = (symrec *)ptr->next)
    if (strcmp (ptr->name,sym_name) == 0)
      return ptr;
  return 0;
@}
@end example

The function @code{yylex} must now recognize variables, numeric values, and
the single-character arithmetic operators.  Strings of alphanumeric
characters with a leading nondigit are recognized as either variables or
functions depending on what the symbol table says about them.

The string is passed to @code{getsym} for look up in the symbol table.  If
the name appears in the table, a pointer to its location and its type
(@code{VAR} or @code{FNCT}) is returned to @code{yyparse}.  If it is not
already in the table, then it is installed as a @code{VAR} using
@code{putsym}.  Again, a pointer and its type (which must be @code{VAR}) is
returned to @code{yyparse}.@refill

No change is needed in the handling of numeric values and arithmetic
operators in @code{yylex}.

@example
#include <ctype.h>
yylex()
@{
  int c;

  /* Ignore whitespace, get first nonwhite character.  */
  while ((c = getchar ()) == ' ' || c == '\t');

  if (c == EOF)
    return 0;

  /* Char starts a number => parse the number.  */
  if (c == '.' || isdigit (c))
    @{
      ungetc (c, stdin);
      scanf ("%lf", &yylval.val);
      return NUM;
    @}

  /* Char starts an identifier => read the name.  */
  if (isalpha (c))
    @{
      symrec *s;
      static char *symbuf = 0;
      static int length = 0;
      int i;

      /* Initially make the buffer long enough
         for a 40-character symbol name.  */
      if (length == 0)
        length = 40, symbuf = (char *)malloc (length + 1);

      i = 0;
      do
        @{
          /* If buffer is full, make it bigger.  */
          if (i == length)
            @{
              length *= 2;
              symbuf = (char *)realloc (symbuf, length + 1);
            @}
          /* Add this character to the buffer.  */
          symbuf[i++] = c;
          /* Get another character.  */
          c = getchar ();
        @}
      while (c != EOF && isalnum (c));

      ungetc (c, stdin);
      symbuf[i] = '\0';

      s = getsym (symbuf);
      if (s == 0)
        s = putsym (symbuf, VAR);
      yylval.tptr = s;
      return s->type;
    @}

  /* Any other character is a token by itself.  */
  return c;
@}
@end example

This program is both powerful and flexible. You may easily add new
functions, and it is a simple job to modify this code to install predefined
variables such as @code{pi} or @code{e} as well.

@node Exercises,, Multi-function calc, Examples
@section Exercises
@cindex exercises

@enumerate
@item
Add some new functions from @file{math.h} to the initialization list.

@item
Add another array that contains constants and their values.  Then
modify @code{init_table} to add these constants to the symbol table.
It will be easiest to give the constants type @code{VAR}.

@item
Make the program report an error if the user refers to an
uninitialized variable in any way except to store a value in it.
@end enumerate

@node Grammar File, Interface, Examples, Top
@chapter Bison Grammar Files

Bison takes as input a context-free grammar specification and produces a
C-language function that recognizes correct instances of the grammar.

The Bison grammar input file conventionally has a name ending in @samp{.y}.

@menu
* Grammar Outline::    Overall layout of the grammar file.
* Symbols::            Terminal and nonterminal symbols.
* Rules::              How to write grammar rules.
* Recursion::	       Writing recursive rules.
* Semantics::          Semantic values and actions.
* Declarations::       All kinds of Bison declarations are described here.
* Multiple Parsers::   Putting more than one Bison parser in one program.
@end menu

@node Grammar Outline, Symbols, Grammar File, Grammar File
@section Outline of a Bison Grammar

A Bison grammar file has four main sections, shown here with the
appropriate delimiters:

@example
%@{
@var{C declarations}
%@}

@var{Bison declarations}

%%
@var{Grammar rules}
%%

@var{Additional C code}
@end example

Comments enclosed in @samp{/* @dots{} */} may appear in any of the sections.

@menu
* C Declarations::      Syntax and usage of the C declarations section.
* Bison Declarations::  Syntax and usage of the Bison declarations section.
* Grammar Rules::       Syntax and usage of the grammar rules section.
* C Code::              Syntax and usage of the additional C code section.
@end menu

@node C Declarations, Bison Declarations, Grammar Outline, Grammar Outline
@subsection The C Declarations Section
@cindex C declarations section
@cindex declarations, C

The @var{C declarations} section contains macro definitions and
declarations of functions and variables that are used in the actions in the
grammar rules.  These are copied to the beginning of the parser file so
that they precede the definition of @code{yylex}.  You can use
@samp{#include} to get the declarations from a header file.  If you don't
need any C declarations, you may omit the @samp{%@{} and @samp{%@}}
delimiters that bracket this section.

@node Bison Declarations, Grammar Rules, C Declarations, Grammar Outline
@subsection The Bison Declarations Section
@cindex Bison declarations section (introduction)
@cindex declarations section, Bison (introduction)

The @var{Bison declarations} section contains declarations that define
terminal and nonterminal symbols, specify precedence, and so on.
In some simple grammars you may not need any declarations.
@xref{Declarations}.

@node Grammar Rules,C Code, Bison Declarations, Grammar Outline
@subsection The Grammar Rules Section
@cindex grammar rules section
@cindex rules section for grammar

The @dfn{grammar rules} section contains one or more Bison grammar
rules, and nothing else.  @xref{Rules}.

There must always be at least one grammar rule, and the first
@samp (which precedes the grammar rules) may never be omitted even
if it is the first thing in the file.

@node C Code,, Grammar Rules, Grammar Outline
@subsection The Additional C Code Section
@cindex additional C code section
@cindex C code, section for additional

The @var{additional C code} section is copied verbatim to the end of
the parser file, just as the @var{C declarations} section is copied to
the beginning.  This is the most convenient place to put anything
that you want to have in the parser file but which need not come before
the definition of @code{yylex}.  For example, the definitions of
@code{yylex} and @code{yyerror} often go here.  @xref{Interface}.

If the last section is empty, you may omit the @samp that separates it
from the grammar rules.

The Bison parser itself contains many static variables whose names start
with @samp{yy} and many macros whose names start with @samp{YY}.  It is a
good idea to avoid using any such names (except those documented in this
manual) in the additional C code section of the grammar file.

@node Symbols, Rules, Grammar Outline, Grammar File
@section Symbols, Terminal and Nonterminal
@cindex nonterminal symbol
@cindex terminal symbol
@cindex token type
@cindex symbol

@dfn{Symbols} in Bison grammars represent the grammatical classifications
of the language.

A @dfn{terminal symbol} (also known as a @dfn{token type}) represents a
class of syntactically equivalent tokens.  You use the symbol in grammar
rules to mean that a token in that class is allowed.  The symbol is
represented in the Bison parser by a numeric code, and the @code{yylex}
function returns a token type code to indicate what kind of token has been
read.  You don't need to know what the code value is; you can use the
symbol to stand for it.

A @dfn{nonterminal symbol} stands for a class of syntactically equivalent
groupings.  The symbol name is used in writing grammar rules.  By convention,
it should be all lower case.

Symbol names can contain letters, digits (not at the beginning),
underscores and periods.  Periods make sense only in nonterminals.

There are two ways of writing terminal symbols in the grammar:

@itemize @bullet
@item
A @dfn{named token type} is written with an identifier, like an
identifier in C.  By convention, it should be all upper case.  Each
such name must be defined with a Bison declaration such as
@code{%token}.  @xref{Token Decl}.

@item
@cindex character token
@cindex literal token
@cindex single-character literal
A @dfn{character token type} (or @dfn{literal token}) is written in
the grammar using the same syntax used in C for character constants;
for example, @code{'+'} is a character token type.  A character token
type doesn't need to be declared unless you need to specify its
semantic value data type (@pxref{Value Type}), associativity, or
precedence (@pxref{Precedence}).

By convention, a character token type is used only to represent a
token that consists of that particular character.  Thus, the token
type @code{'+'} is used to represent the character @samp{+} as a
token.  Nothing enforces this convention, but if you depart from it,
your program will confuse other readers.

All the usual escape sequences used in character literals in C can be
used in Bison as well, but you must not use the null character as a
character literal because its ASCII code, zero, is the code
@code{yylex} returns for end-of-input (@pxref{Calling Convention}).
@end itemize

How you choose to write a terminal symbol has no effect on its
grammatical meaning.  That depends only on where it appears in rules and
on when the parser function returns that symbol.

The value returned by @code{yylex} is always one of the terminal symbols
(or 0 for end-of-input).  Whichever way you write the token type in the
grammar rules, you write it the same way in the definition of @code{yylex}.
The numeric code for a character token type is simply the ASCII code for
the character, so @code{yylex} can use the identical character constant to
generate the requisite code.  Each named token type becomes a C macro in
the parser file, so @code{yylex} can use the name to stand for the code.
(This is why periods don't make sense in terminal symbols.)  @xref{Calling
Convention}.

If @code{yylex} is defined in a separate file, you need to arrange for the
token-type macro definitions to be available there.  Use the @samp{-d}
option when you run Bison, so that it will write these macro definitions
into a separate header file @file{@var{name}.tab.h} which you can include
in the other source files that need it.  @xref{Invocation}.

The symbol @code{error} is a terminal symbol reserved for error recovery
(@pxref{Error Recovery}); you shouldn't use it for any other purpose.
In particular, @code{yylex} should never return this value.

@node Rules, Recursion, Symbols, Grammar File
@section Syntax of Grammar Rules
@cindex rule syntax
@cindex grammar rule syntax
@cindex syntax of grammar rules

A Bison grammar rule has the following general form:

@example
@var{result}: @var{components}@dots{}
        ;
@end example

@noindent
where @var{result} is the nonterminal symbol that this rule describes
and @var{components} are various terminal and nonterminal symbols that
are put together by this rule (@pxref{Symbols}).  For example,

@example
exp:      exp '+' exp
        ;
@end example

@noindent
says that two groupings of type @code{exp}, with a @samp{+} token in between,
can be combined into a larger grouping of type @code{exp}.

Whitespace in rules is significant only to separate symbols.  You can add
extra whitespace as you wish.

Scattered among the components can be @var{actions} that determine
the semantics of the rule.  An action looks like this:

@example
@{@var{C statements}@}
@end example

@noindent
Usually there is only one action and it follows the components.
@xref{Actions}.

@findex |
Multiple rules for the same @var{result} can be written separately or can
be joined with the vertical-bar character @samp{|} as follows:

@ifinfo
@example
@var{result}:   @var{rule1-components}@dots{}
        | @var{rule2-components}@dots{}
        @dots{}
        ;
@end example
@end ifinfo
@iftex
@example
@var{result}:    @var{rule1-components}@dots{}
        | @var{rule2-components}@dots{}
        @dots{}
        ;
@end example
@end iftex

@noindent
They are still considered distinct rules even when joined in this way.

If @var{components} in a rule is empty, it means that @var{result} can
match the empty string.  For example, here is how to define a
comma-separated sequence of zero or more @code{exp} groupings:

@example
expseq:   /* empty */
        | expseq1
        ;

expseq1:  exp
        | expseq1 ',' exp
        ;
@end example

@noindent
It is customary to write a comment @samp{/* empty */} in each rule
with no components.

@node Recursion, Semantics, Rules, Grammar File
@section Recursive Rules
@cindex recursive rule

A rule is called @dfn{recursive} when its @var{result} nonterminal appears
also on its right hand side.  Nearly all Bison grammars need to use
recursion, because that is the only way to define a sequence of any number
of somethings.  Consider this recursive definition of a comma-separated
sequence of one or more expressions:

@example
expseq1:  exp
        | expseq1 ',' exp
        ;
@end example

@cindex left recursion
@cindex right recursion
@noindent
Since the recursive use of @code{expseq1} is the leftmost symbol in the
right hand side, we call this @dfn{left recursion}.  By contrast, here
the same construct is defined using @dfn{right recursion}:

@example
expseq1:  exp
        | exp ',' expseq1
        ;
@end example

@noindent
Any kind of sequence can be defined using either left recursion or right
recursion, but you should always use left recursion, because it can parse a
sequence of any number of elements with bounded stack space.  Right
recursion uses up space on the Bison stack in proportion to the number of
elements in the sequence, because all the elements must be shifted onto the
stack before the rule can be applied even once.  @xref{Algorithm},
for further explanation of this.

@cindex mutual recursion
@dfn{Indirect} or @dfn{mutual} recursion occurs when the result of the
rule does not appear directly on its right hand side, but does appear
in rules for other nonterminals which do appear on its right hand
side.  For example:

@example
expr:     primary
        | primary '+' primary
        ;

primary:  constant
        | '(' expr ')'
        ;
@end example

@noindent
defines two mutually-recursive nonterminals, since each refers to the
other.

@node Semantics, Declarations, Recursion, Grammar File
@section The Semantics of the Language
@cindex language semantics
@cindex semantics of the language

The grammar rules for a language determine only the syntax.  The semantics
are determined by the semantic values associated with various tokens and
groupings, and by the actions taken when various groupings are recognized.

For example, the calculator calculates properly because the value
associated with each expression is the proper number; it adds properly
because the action for the grouping @w{@samp{@var{x} + @var{y}}} is to add
the numbers associated with @var{x} and @var{y}.

@menu
* Value Type::       Specifying one data type for all semantic values.
* Multiple Types::   Specifying several alternative data types.
* Actions::          An action is the semantic definition of a grammar rule.
* Action Types::     Specifying data types for actions to operate on.
* Mid-Rule Actions:: Most actions go at the end of a rule.
                      This says when, why and how to use the exceptional
                      action in the middle of a rule.
@end menu

@node Value Type, Multiple Types, Semantics, Semantics
@subsection The Data Types of Semantic Values
@cindex semantic value type
@cindex value type, semantic
@cindex data types of semantic values

In a simple program it may be sufficient to use the same data type for
the semantic values of all language constructs.  This was true in the
RPN and infix calculator examples (@pxref{RPN Calc}).

Bison's default is to use type @code{int} for all semantic values.  To
specify some other type, define @code{YYSTYPE} as a macro, like this:

@example
#define YYSTYPE double
@end example

@noindent
This macro definition must go in the C declarations section of the grammar
file (@pxref{Grammar Outline}).

@node Multiple Types, Actions, Value Type, Semantics
@subsection More Than One Type for Semantic Values

In most programs, you will need different data types for different kinds
of tokens and groupings.  For example, a numeric constant may need type
@code{int} or @code{long}, while a string constant needs type @code{char *},
and an identifier might need a pointer to an entry in the symbol table.

To use more than one data type for semantic values in one parser, Bison
requires you to do two things:

@itemize @bullet
@item
Specify the entire collection of possible data types, with the
@code{%union} Bison declaration (@pxref{Union Decl}).

@item
Choose one of those types for each symbol (terminal or nonterminal)
for which semantic values are used.  This is done for tokens with the
@code{%token} Bison declaration (@pxref{Token Decl}) and for groupings
with the @code{%type} Bison declaration (@pxref{Type Decl}).
@end itemize

@node Actions, Action Types, Multiple Types, Semantics
@subsection Actions
@cindex action
@vindex $$
@vindex $@var{n}

An action accompanies a syntactic rule and contains C code to be executed
each time an instance of that rule is recognized.  The task of most actions
is to compute a semantic value for the grouping built by the rule from the
semantic values associated with tokens or smaller groupings.

An action consists of C statements surrounded by braces, much like a
compound statement in C.  It can be placed at any position in the rule; it
is executed at that position.  Most rules have just one action at the end
of the rule, following all the components.  Actions in the middle of a rule
are tricky and used only for special purposes (@pxref{Mid-Rule Actions}).

The C code in an action can refer to the semantic values of the components
matched by the rule with the construct @code{$@var{n}}, which stands for
the value of the @var{n}th component.  The semantic value for the grouping
being constructed is @code{$$}.  (Bison translates both of these constructs
into array element references when it copies the actions into the parser
file.)

Here is a typical example:

@example
exp:    @dots{}
        | exp '+' exp
            @{ $$ = $1 + $3; @}
@end example

@noindent
This rule constructs an @code{exp} from two smaller @code{exp} groupings
connected by a plus-sign token.  In the action, @code{$1} and @code{$3}
refer to the semantic values of the two component @code{exp} groupings,
which are the first and third symbols on the right hand side of the rule.
The sum is stored into @code{$$} so that it becomes the semantic value of
the addition-expression just recognized by the rule.  If there were a
useful semantic value associated with the @samp{+} token, it could be
referred to as @code{$2}.@refill

@code{$@var{n}} with @var{n} zero or negative is allowed for reference
to tokens and groupings on the stack @emph{before} those that match the
current rule.  This is a very risky practice, and to use it reliably
you must be certain of the context in which the rule is applied.  Here
is a case in which you can use this reliably:

@example
foo:      expr bar '+' expr  @{ @dots{} @}
        | expr bar '-' expr  @{ @dots{} @}
        ;

bar:      /* empty */
        @{ previous_expr = $0; @}
        ;
@end example

As long as @code{bar} is used only in the fashion shown here, @code{$0}
always refers to the @code{expr} which precedes @code{bar} in the
definition of @code{foo}.

@node Action Types, Mid-Rule Actions, Actions, Semantics
@subsection Data Types of Values in Actions
@cindex action data types
@cindex data types in actions

If you have chosen a single data type for semantic values, the @code{$$}
and @code{$@var{n}} constructs always have that data type.

If you have used @code{%union} to specify a variety of data types, then you
must declare a choice among these types for each terminal or nonterminal
symbol that can have a semantic value.  Then each time you use @code{$$} or
@code{$@var{n}}, its data type is determined by which symbol it refers to
in the rule.  In this example,@refill

@example
exp:    @dots{}
        | exp '+' exp
            @{ $$ = $1 + $3; @}
@end example

@noindent
@code{$3} and @code{$$} refer to instances of @code{exp}, so they all have
the data type declared for the nonterminal symbol @code{exp}.  If @code{$2}
were used, it would have the data type declared for the terminal symbol
@code{'+'}, whatever that might be.@refill

Alternatively, you can specify the data type when you refer to the value,
by inserting @samp{<@var{type}>} after the @samp{$} at the beginning of the
reference.  For example, if you have defined types as shown here:

@example
%union @{
  int itype;
  double dtype;
@}
@end example

@noindent
then you can write @code{$<itype>1} to refer to the first subunit of the
rule as an integer, or @code{$<dtype>1} to refer to it as a double.

@node Mid-Rule Actions,, Action Types, Semantics
@subsection Actions in Mid-Rule
@cindex actions in mid-rule
@cindex mid-rule actions

Occasionally it is useful to put an action in the middle of a rule.
These actions are written just like usual end-of-rule actions, but they
are executed before the parser even recognizes the following components.

A mid-rule action may refer to the components preceding it using
@code{$@var{n}}, but it may not refer to subsequent components because
it is run before they are parsed.

The mid-rule action itself counts as one of the components of the rule.
This makes a difference when there is another action later in the same rule
(and usually there is another at the end): you have to count the actions
along with the symbols when working out which number @var{n} to use in
@code{$@var{n}}.

The mid-rule action can also have a semantic value.  This can be set within
that action by an assignment to @code{$$}, and can referred to by later
actions using @code{$@var{n}}.  Since there is no symbol to name the
action, there is no way to declare a data type for the value in advance, so
you must use the @samp{$<@dots{}>} construct to specify a data type each
time you refer to this value.

Here is an example from a hypothetical compiler, handling a @code{let}
statement that looks like @samp{let (@var{variable}) @var{statement}} and
serves to create a variable named @var{variable} temporarily for the
duration of @var{statement}.  To parse this construct, we must put
@var{variable} into the symbol table while @var{statement} is parsed, then
remove it afterward.  Here is how it is done:

@example
stmt:   LET '(' var ')'
                @{ $<context>$ = push_context ();
                  declare_variable ($3); @}
        stmt    @{ $$ = $6;
                  pop_context ($<context>5); @}
@end example

@noindent
As soon as @samp{let (@var{variable})} has been recognized, the first
action is run.  It saves a copy of the current semantic context (the
list of accessible variables) as its semantic value, using alternative
@code{context} in the data-type union.  Then it calls
@code{declare_variable} to add the new variable to that list.  Once the
first action is finished, the embedded statement @code{stmt} can be
parsed.  Note that the mid-rule action is component number 5, so the
@samp{stmt} is component number 6.

After the embedded statement is parsed, its semantic value becomes the
value of the entire @code{let}-statement.  Then the semantic value from the
earlier action is used to restore the prior list of variables.  This
removes the temporary @code{let}-variable from the list so that it won't
appear to exist while the rest of the program is parsed.

Taking action before a rule is completely recognized often leads to
conflicts since the parser must commit to a parse in order to execute the
action.  For example, the following two rules, without mid-rule actions,
can coexist in a working parser because the parser can shift the open-brace
token and look at what follows before deciding whether there is a
declaration or not:

@example
compound: '@{' declarations statements '@}'
        | '@{' statements '@}'
        ;
@end example

@noindent
But when we add a mid-rule action as follows, the rules become nonfunctional:

@example
compound: @{ prepare_for_local_variables (); @}
          '@{' declarations statements '@}'
@group
        | '@{' statements '@}'
        ;
@end group
@end example

@noindent
Now the parser is forced to decide whether to run the mid-rule action
when it has read no farther than the open-brace.  In other words, it
must commit to using one rule or the other, without sufficient
information to do it correctly.  (The open-brace token is what is called
the @dfn{look-ahead} token at this time, since the parser is still
deciding what to do about it.  @xref{Look-Ahead}.)

You might think that you could correct the problem by putting identical
actions into the two rules, like this:

@example
compound: @{ prepare_for_local_variables (); @}
          '@{' declarations statements '@}'
        | @{ prepare_for_local_variables (); @}
          '@{' statements '@}'
        ;
@end example

@noindent
But this does not help, because Bison does not realize that the two actions
are identical.  (Bison never tries to understand the C code in an action.)

If the grammar is such that a declaration can be distinguished from a
statement by the first token (which is true in C), then one solution which
does work is to put the action after the open-brace, like this:

@example
compound: '@{' @{ prepare_for_local_variables (); @}
          declarations statements '@}'
        | '@{' statements '@}'
        ;
@end example

@noindent
Now the first token of the following declaration or statement,
which would in any case tell Bison which rule to use, can still do so.

Another solution is to bury the action inside a nonterminal symbol which
serves as a subroutine:

@example
subroutine: /* empty */
          @{ prepare_for_local_variables (); @}
        ;

compound: subroutine
          '@{' declarations statements '@}'
        | subroutine
          '@{' statements '@}'
        ;
@end example

@noindent
Now Bison can execute the action in the rule for @code{subroutine} without
deciding which rule for @code{compound} it will eventually use.  Note that
the action is now at the end of its rule.  Any mid-rule action can be
converted to an end-of-rule action in this way, and this is what Bison
actually does to implement mid-rule actions.

@node Declarations, Multiple Parsers, Semantics, Grammar File
@section Bison Declarations
@cindex declarations, Bison
@cindex Bison declarations

The @dfn{Bison declarations} section of a Bison grammar defines the symbols
used in formulating the grammar and the data types of semantic values.
@xref{Symbols}.

All token type names (but not single-character literal tokens such as
@code{'+'} and @code{'*'}) must be declared.  Nonterminal symbols must be
declared if you need to specify which data type to use for the semantic
value (@pxref{Multiple Types}).

The first rule in the file also specifies the start symbol, by default.
If you want some other symbol to be the start symbol, you must declare
it explicitly (@pxref{Language and Grammar}).

@menu
* Token Decl::       Declaring terminal symbols.
* Precedence Decl::  Declaring terminals with precedence and associativity.
* Union Decl::       Declaring the set of all semantic value types.
* Type Decl::        Declaring the choice of type for a nonterminal symbol.
* Expect Decl::      Suppressing warnings about shift/reduce conflicts.
* Start Decl::       Specifying the start symbol.
* Pure Decl::        Requesting a reentrant parser.
* Decl Summary::     Table of all Bison declarations.
@end menu

@node Token Decl, Precedence Decl, Declarations, Declarations
@subsection Declaring Token Type Names
@cindex declaring token type names
@cindex token type names, declaring
@findex %token

The basic way to declare a token type name (terminal symbol) is as follows:

@example
%token @var{name}
@end example

Bison will convert this into a @code{#define} directive in
the parser, so that the function @code{yylex} (if it is in this file)
can use the name @var{name} to stand for this token type's code.

Alternatively you can use @code{%left}, @code{%right}, or @code{%nonassoc}
instead of @code{%token}, if you wish to specify precedence.
@xref{Precedence Decl}.

You can explicitly specify the numeric code for a token type by appending
an integer value in the field immediately following the token name:

@example
%token NUM 300
@end example

@noindent
It is generally best, however, to let Bison choose the numeric codes for
all token types.  Bison will automatically select codes that don't conflict
with each other or with ASCII characters.

In the event that the stack type is a union, you must augment the
@code{%token} or other token declaration to include the data type
alternative delimited by angle-brackets (@pxref{Multiple Types}).  For
example:

@example
%union @{              /* define stack type */
  double val;
  symrec *tptr;
@}
%token <val> NUM      /* define token NUM and its type */
@end example

@node Precedence Decl, Union Decl, Token Decl, Declarations
@subsection Declaring Operator Precedence
@cindex declaring operator precedence
@cindex operator precedence, declaring

Use the @code{%left}, @code{%right} or @code{%nonassoc} declaration to
declare a token and specify its precedence and associativity, all at
once.  These are called @dfn{precedence declarations}.
@xref{Precedence}, for general information on operator precedence.

The syntax of a precedence declaration is the same as that of
@code{%token}: either

@example
%left @var{symbols}@dots{}
@end example

@noindent
or

@example
%left <@var{type}> @var{symbols}@dots{}
@end example

And indeed any of these declarations serves the purposes of @code{%token}.
But in addition, they specify the associativity and relative precedence for
all the @var{symbols}:

@itemize @bullet
@item
The associativity of an operator @var{op} determines how repeated uses
of the operator nest: whether @samp{@var{x} @var{op} @var{y} @var{op}
@var{z}} is parsed by grouping @var{x} with @var{y} first or by
grouping @var{y} with @var{z} first.  @code{%left} specifies
left-associativity (grouping @var{x} with @var{y} first) and
@code{%right} specifies right-associativity (grouping @var{y} with
@var{z} first).  @code{%nonassoc} specifies no associativity, which
means that @samp{@var{x} @var{op} @var{y} @var{op} @var{z}} is
considered a syntax error.

@item
The precedence of an operator determines how it nests with other operators.
All the tokens declared in a single precedence declaration have equal
precedence and nest together according to their associativity.
When two tokens declared in different precedence declarations associate,
the one declared later has the higher precedence and is grouped first.
@end itemize

@node Union Decl, Type Decl, Precedence Decl, Declarations
@subsection Declaring the Collection of Value Types
@cindex declaring value types
@cindex value types, declaring
@findex %union

The @code{%union} declaration specifies the entire collection of possible
data types for semantic values.  The keyword @code{%union} is followed by a
pair of braces containing the same thing that goes inside a @code{union} in
C.  For example:

@example
%union @{
  double val;
  symrec *tptr;
@}
@end example

@noindent
This says that the two alternative types are @code{double} and @code{symrec
*}.  They are given names @code{val} and @code{tptr}; these names are used
in the @code{%token} and @code{%type} declarations to pick one of the types
for a terminal or nonterminal symbol (@pxref{Type Decl}).

Note that, unlike making a @code{union} declaration in C, you do not write
a semicolon after the closing brace.

@node Type Decl, Expect Decl, Union Decl, Declarations
@subsection Declaring Value Types of Nonterminal Symbols
@cindex declaring value types, nonterminals
@cindex value types, nonterminals, declaring
@findex %type

@noindent
When you use @code{%union} to specify multiple value types, you must
declare the value type of each nonterminal symbol for which values are
used.  This is done with a @code{%type} declaration, like this:

@example
%type <@var{type}> @var{nonterminal}@dots{}
@end example

@noindent
Here @var{nonterminal} is the name of a nonterminal symbol, and @var{type}
is the name given in the @code{%union} to the alternative that you want
(@pxref{Union Decl}).  You can give any number of nonterminal symbols in
the same @code{%type} declaration, if they have the same value type.  Use
spaces to separate the symbol names.

@node Expect Decl, Start Decl, Type Decl, Declarations
@subsection Preventing Warnings about Conflicts
@cindex preventing warnings about conflicts
@cindex warnings, preventing
@cindex conflicts, preventing warnings of
@findex %expect

Bison normally warns if there are any conflicts in the grammar
(@pxref{Shift/Reduce}), but most real grammars have harmless shift/reduce
conflicts which are resolved in a predictable way and would be difficult to
eliminate.  It is desirable to suppress the warning about these conflicts
unless the number of conflicts changes.  You can do this with the
@code{%expect} declaration.

The declaration looks like this:

@example
%expect @var{n}
@end example

Here @var{n} is a decimal integer.  The declaration says there should be no
warning if there are @var{n} shift/reduce conflicts and no reduce/reduce
conflicts.  The usual warning is given if there are either more or fewer
conflicts, or if there are any reduce/reduce conflicts.

In general, using @code{%expect} involves these steps:

@itemize @bullet
@item
Compile your grammar without @code{%expect}.  Use the @samp{-v} option
to get a verbose list of where the conflicts occur.  Bison will also
print the number of conflicts.

@item
Check each of the conflicts to make sure that Bison's default
resolution is what you really want.  If not, rewrite the grammar and
go back to the beginning.

@item
Add an @code{%expect} declaration, copying the number @var{n} from the
number which Bison printed.
@end itemize

Now Bison will stop annoying you about the conflicts you have checked, but
it will warn you again if changes in the grammer result in additional
conflicts.

@node Start Decl, Pure Decl, Expect Decl, Declarations
@subsection Declaring the Start-Symbol
@cindex declaring the start-symbol
@cindex start-symbol, declaring
@findex %start

Bison assumes by default that the start symbol for the grammar is the first
nonterminal specified in the grammar specification section.  The programmer
may override this restriction with the @code{%start} declaration as follows:

@example
%start @var{symbol}
@end example

@node Pure Decl, Decl Summary, Start Decl, Declarations
@subsection Requesting a Pure (Reentrant) Parser
@cindex reentrant parser
@cindex pure parser
@findex %pure_parser

A @dfn{reentrant} program is one which does not alter in the course of
execution; in other words, it consists entirely of @dfn{pure} (read-only)
code.  Reentrancy is important whenever asynchronous execution is possible;
for example, a nonreentrant program may not be safe to call from a signal
handler.  In systems with multiple threads of control, a nonreentrant
program must be called only within interlocks.

The Bison parser is not normally a reentrant program, because it uses
statically allocated variables for communication with @code{yylex}.  These
variables include @code{yylval} and @code{yylloc}.

The Bison declaration @code{%pure_parser} says that you want the parser
to be reentrant.  It looks like this:

@example
%pure_parser
@end example

The effect is that the the two communication variables become local
variables in @code{yyparse}, and a different calling convention is used for
the lexical analyzer function @code{yylex}.  @xref{Pure Calling}, for the
details of this.  The variable @code{yynerrs} also becomes local in
@code{yyparse} (@pxref{Error Reporting}).  The convention for calling
@code{yyparse} itself is unchanged.

@node Decl Summary,, Pure Decl, Declarations
@subsection Bison Declaration Summary
@cindex Bison declaration summary
@cindex declaration summary
@cindex summary, Bison declaration

Here is a summary of all Bison declarations:

@table @code
@item %union
Declare the collection of data types that semantic values may have
(@pxref{Union Decl}).

@item %token
Declare a terminal symbol (token type name) with no precedence
or associativity specified (@pxref{Token Decl}).

@item %right
Declare a terminal symbol (token type name) that is right-associative
(@pxref{Precedence Decl}).

@item %left
Declare a terminal symbol (token type name) that is left-associative
(@pxref{Precedence Decl}).

@item %nonassoc
Declare a terminal symbol (token type name) that is nonassociative
(using it in a way that would be associative is a syntax error)
(@pxref{Precedence Decl}).

@item %type
Declare the type of semantic values for a nonterminal symbol
(@pxref{Type Decl}).

@item %start
Specify the grammar's start symbol (@pxref{Start Decl}).

@item %expect
Declare the expected number of shift-reduce conflicts
(@pxref{Expect Decl}).

@item %pure_parser
Request a pure (reentrant) parser program (@pxref{Pure Decl}).
@end table

@node Multiple Parsers,, Declarations, Grammar File
@section Multiple Parsers in the Same Program

Most programs that use Bison parse only one language and therefore contain
only one Bison parser.  But what if you want to parse more than one
language with the same program?  Here is what you must do:

@itemize @bullet
@item
Make each parser a pure parser (@pxref{Pure Decl}).  This gets rid of
global variables such as @code{yylval} which would otherwise conflict
between the various parsers, but it requires an alternate calling
convention for @code{yylex} (@pxref{Pure Calling}).

@item
In each grammar file, define @code{yyparse} as a macro, expanding
into the name you want for that parser.  Put this definition in
the C declarations section (@pxref{C Declarations}).  For example:

@example
%@{
#define yyparse parse_algol
%@}
@end example

@noindent
Then use the expanded name @code{parse_algol} in other source files to
call this parser.

@item
If you want different lexical analyzers for each grammar, you can
define @code{yylex} as a macro, just like @code{yyparse}.  Use
the expanded name when you define @code{yylex} in another source
file.

If you define @code{yylex} in the grammar file itself, simply
make it static, like this:

@example
%@{
static int yylex ();
%@}
%%
@dots{} @var{grammar rules} @dots{}
%%
static int
yylex (yylvalp, yyllocp)
     YYSTYPE *yylvalp;
     YYLTYPE *yyllocp;
@{ @dots{} @}
@end example

@item
If you want a different @code{yyerror} function for each grammar,
you can use the same methods that work for @code{yylex}.
@end itemize

@node Interface, Algorithm, Grammar File, Top
@chapter Parser C-Language Interface
@cindex C-language interface
@cindex interface

The Bison parser is actually a C function named @code{yyparse}.  Here we
describe the interface conventions of @code{yyparse} and the other
functions that it needs to use.

Keep in mind that the parser uses many C identifiers starting with
@samp{yy} and @samp{YY} for internal purposes.  If you use such an
identifier (aside from those in this manual) in an action or in additional
C code in the grammar file, you are likely to run into trouble.

@menu
* Parser Function:: How to call @code{yyparse} and what it returns.
* Lexical::         You must supply a function @code{yylex} which reads tokens.
* Error Reporting:: You must supply a function @code{yyerror}.
* Action Features:: Special features for use in actions.
@end menu

@node Parser Function, Lexical, Interface, Interface
@section The Parser Function @code{yyparse}
@findex yyparse

You call the function @code{yyparse} to cause parsing to occur.  This
function reads tokens, executes actions, and ultimately returns when it
encounters end-of-input or an unrecoverable syntax error.  You can also
write an action which directs @code{yyparse} to return immediately without
reading further.

The value returned by @code{yyparse} is 0 if parsing was successful (return
is due to end-of-input).

The value is 1 if parsing failed (return is due to a syntax error).

In an action, you can cause immediate return from @code{yyparse} by using
these macros:

@table @code
@item YYACCEPT
@findex YYACCEPT
Return immediately with value 0 (to report success).

@item YYABORT
@findex YYABORT
Return immediately with value 1 (to report failure).
@end table

@node Lexical, Error Reporting, Parser Function, Interface
@section The Lexical Analyzer Function @code{yylex}
@findex yylex
@cindex lexical analyzer

The @dfn{lexical analyzer} function, @code{yylex}, recognizes tokens from
the input stream and returns them to the parser.  Bison does not create
this function automatically; you must write it so that @code{yyparse} can
call it.  The function is sometimes referred to as a lexical scanner.

In simple programs, @code{yylex} is often defined at the end of the Bison
grammar file.  If @code{yylex} is defined in a separate source file, you
need to arrange for the token-type macro definitions to be available there.
To do this, use the @samp{-d} option when you run Bison, so that it will
write these macro definitions into a separate header file
@file{@var{name}.tab.h} which you can include in the other source files
that need it.  @xref{Invocation}.@refill

@menu
* Calling Convention::   How @code{yyparse} calls @code{yylex}.
* Token Values::         How @code{yylex} must return the semantic value
                           of the token it has read.
* Token Positions::      How @code{yylex} must return the text position
                           (line number, etc.) of the token, if the
                           actions want that.
* Pure Calling::         How the calling convention differs
                           in a pure parser (@pxref{Pure Decl}).
@end menu

@node Calling Convention, Token Values, Lexical, Lexical
@subsection Calling Convention for @code{yylex}

The value that @code{yylex} returns must be the numeric code for the type
of token it has just found, or 0 for end-of-input.

When a token is referred to in the grammar rules by a name, that name
in the parser file becomes a C macro whose definition is the proper
numeric code for that token type.  So @code{yylex} can use the name
to indicate that type.  @xref{Symbols}.

When a token is referred to in the grammar rules by a character literal,
the numeric code for that character is also the code for the token type.
So @code{yylex} can simply return that character code.  The null character
must not be used this way, because its code is zero and that is what
signifies end-of-input.

Here is an example showing these things:

@example
yylex()
@{
  @dots{}
  if (c == EOF)     /* Detect end of file.  */
    return 0;
  @dots{}
  if (c == '+' || c == '-')
    return c;      /* Assume token type for `+' is '+'.  */
  @dots{}
  return INT;      /* Return the type of the token.  */
  @dots{}
@}
@end example

@noindent
This interface has been designed so that the output from the @code{lex}
utility can be used without change as the definition of @code{yylex}.

@node Token Values, Token Positions, Calling Convention, Lexical
@subsection Returning Semantic Values of Tokens

@vindex yylval
In an ordinary (nonreentrant) parser, the semantic value of the token must
be stored into the global variable @code{yylval}.  When you are using
just one data type for semantic values, @code{yylval} has that type.
Thus, if the type is @code{int} (the default), you might write this in
@code{yylex}:

@example
  @dots{}
  yylval = value;  /* Put value onto Bison stack.  */
  return INT;      /* Return the type of the token.  */
  @dots{}
@end example

When you are using multiple data types, @code{yylval}'s type is a union
made from the @code{%union} declaration (@pxref{Union Decl}).  So when
you store a token's value, you must use the proper member of the union.
If the @code{%union} declaration looks like this:

@example
%union @{
  int intval;
  double val;
  symrec *tptr;
@}
@end example

@noindent
then the code in @code{yylex} might look like this:

@example
  @dots{}
  yylval.intval = value;  /* Put value onto Bison stack.  */
  return INT;      /* Return the type of the token.  */
  @dots{}
@end example

@node Token Positions, Pure Calling, Token Values, Lexical
@subsection Reporting Textual Positions of Tokens

@vindex yylloc
If you are using the @samp{@@@var{n}}-feature (@pxref{Action Features}) in
actions to keep track of the textual locations of tokens and groupings,
then you must provide this information in @code{yylex}.  The function
@code{yyparse} expects to find the textual location of a token just parsed
in the global variable @code{yylloc}.  So @code{yylex} must store the
proper data in that variable.  The value of @code{yylloc} is a structure
and you need only initialize the members that are going to be used by the
actions.  The four members are called @code{first_line},
@code{first_column}, @code{last_line} and @code{last_column}.  Note that
the use of this feature makes the parser noticeably slower.

@tindex YYLTYPE
The data type of @code{yylloc} has the name @code{YYLTYPE}.

@node Pure Calling,, Token Positions, Lexical
@subsection Calling Convention for Pure Parsers

When you use the Bison declaration @code{%pure_parser} to request a pure,
reentrant parser, the global communication variables @code{yylval} and
@code{yylloc} cannot be used.  (@xref{Pure Decl}.)  In such parsers the
two global variables are replaced by pointers passed as arguments to
@code{yylex}.  You must declare them as shown here, and pass the
information back by storing it through those pointers.

@example
yylex (lvalp, llocp)
     YYSTYPE *lvalp;
     YYLTYPE *llocp;
@{
  @dots{}
  *lvalp = value;  /* Put value onto Bison stack.  */
  return INT;      /* Return the type of the token.  */
  @dots{}
@}
@end example

@node Error Reporting, Action Features, Lexical, Interface
@section The Error Reporting Function @code{yyerror}
@cindex error reporting function
@findex yyerror
@cindex parse error
@cindex syntax error

The Bison parser detects a @dfn{parse error} or @dfn{syntax error}
whenever it reads a token which cannot satisfy any syntax rule.  A
action in the grammar can also explicitly proclaim an error, using the
macro @code{YYERROR} (@pxref{Action Features}).

The Bison parser expects to report the error by calling an error
reporting function named @code{yyerror}, which you must supply.  It is
called by @code{yyparse} whenever a syntax error is found, and it
receives one argument.  For a parse error, the string is always
@w{@code{"parse error"}}.

The parser can detect one other kind of error: stack overflow.  This
happens when the input contains constructions that are very deeply
nested.  It isn't likely you will encounter this, since the Bison
parser extends its stack automatically up to a very large limit.  But
if overflow happens, @code{yyparse} calls @code{yyerror} in the usual
fashion, except that the argument string is @w{@code{"parser stack
overflow"}}.

The following definition suffices in simple programs:

@example
yyerror (s)
     char *s;
@{
@group
  fprintf (stderr, "%s\n", s);
@}
@end group
@end example

After @code{yyerror} returns to @code{yyparse}, the latter will attempt
error recovery if you have written suitable error recovery grammar rules
(@pxref{Error Recovery}).  If recovery is impossible, @code{yyparse} will
immediately return 1.

@vindex yynerrs
The variable @code{yynerrs} contains the number of syntax errors
encountered so far.  Normally this variable is global; but if you
request a pure parser (@pxref{Pure Decl}) then it is a local variable
which only the actions can access.

@node Action Features,, Error Reporting, Interface
@section Special Features for Use in Actions
@cindex summary, action features
@cindex action features summary

Here is a table of Bison constructs, variables and macros that
are useful in actions.

@table @samp
@item $$
Acts like a variable that contains the semantic value for the
grouping made by the current rule.  @xref{Actions}.

@item $@var{n}
Acts like a variable that contains the semantic value for the
@var{n}th component of the current rule.  @xref{Actions}.

@item $<@var{typealt}>$
Like @code{$$} but specifies alternative @var{typealt} in the union
specified by the @code{%union} declaration.  @xref{Action Types}.

@item $<@var{typealt}>@var{n}
Like @code{$@var{n}} but specifies alternative @var{typealt} in the
union specified by the @code{%union} declaration.  @xref{Action
Types}.@refill

@item YYABORT;
Return immediately from @code{yyparse}, indicating failure.
@xref{Parser Function}.

@item YYACCEPT;
Return immediately from @code{yyparse}, indicating success.
@xref{Parser Function}.

@item YYEMPTY
Value stored in @code{yychar} when there is no look-ahead token.

@item YYERROR;
Cause an immediate syntax error.  This causes @code{yyerror} to
be called, and then error recovery begins.  @xref{Error Recovery}.

@item yychar
Variable containing the current look-ahead token.  (In a pure parser,
this is actually a local variable within @code{yyparse}.)  When there
is no look-ahead token, the value @code{YYERROR} is stored here.
@xref{Look-Ahead}.

@item yyclearin;
Discard the current look-ahead token.  This is useful primarily in
error rules.  @xref{Error Recovery}.

@item yyerrok;
Resume generating error messages immediately for subsequent syntax
errors.  This is useful primarily in error rules.  @xref{Error
Recovery}.

@item @@@var{n}
@findex @@@var{n}
Acts like a structure variable containing information on the line
numbers and column numbers of the @var{n}th component of the current
rule.  The structure has four members, like this:

@example
struct @{
  int first_line, last_line;
  int first_column, last_column;
@};
@end example

Thus, to get the starting line number of the third component, use
@samp{@@3.first_line}.

In order for the members of this structure to contain valid information,
you must make @code{yylex} supply this information about each token.
If you need only certain members, then @code{yylex} need only fill in
those members.

The use of this feature makes the parser noticeably slower.
@end table

@node Algorithm, Error Recovery, Interface, Top
@chapter The Algorithm of the Bison Parser
@cindex algorithm of parser
@cindex shifting
@cindex reduction
@cindex parser stack
@cindex stack, parser

As Bison reads tokens, it pushes them onto a stack along with their
semantic values.  The stack is called the @dfn{parser stack}.  Pushing a
token is traditionally called @dfn{shifting}.

For example, suppose the infix calculator has read @samp{1 + 5 *}, with a
@samp{3} to come.  The stack will have four elements, one for each token
that was shifted.

But the stack does not always have an element for each token read.  When
the last @var{n} tokens and groupings shifted match the components of a
grammar rule, they can be combined according to that rule.  This is called
@dfn{reduction}.  Those tokens and groupings are replaced on the stack by a
single grouping whose symbol is the result (left hand side) of that rule.
Running the rule's action is part of the process of reduction, because this
is what computes the semantic value of the resulting grouping.

For example, if the infix calculator's parser stack contains this:

@example
1 + 5 * 3
@end example

@noindent
and the next input token is a newline character, then the last three
elements can be reduced to 15 via the rule:

@example
expr: expr '*' expr;
@end example

@noindent
Then the stack contains just these three elements:

@example
1 + 15
@end example

@noindent
At this point, another reduction can be made, resulting in the single value
16.  Then the newline token can be shifted.

The parser tries, by shifts and reductions, to reduce the entire input down
to a single grouping whose symbol is the grammar's start-symbol
(@pxref{Language and Grammar}).

This kind of parser is known in the literature as a bottom-up parser.

@menu
* Look-Ahead::          Parser looks one token ahead when deciding what to do.
* Shift/Reduce::        Conflicts: when either shifting or reduction is valid.
* Precedence::          Operator precedence works by resolving conflicts.
* Contextual Precedence:: When an operator's precedence depends on context.
* Parser States::       The parser is a finite-state-machine with stack.
* Reduce/Reduce::       When two rules are applicable in the same situation.
@end menu

@node Look-Ahead, Shift/Reduce, Algorithm, Algorithm
@section Look-Ahead Tokens
@cindex look-ahead token

The Bison parser does @emph{not} always reduce immediately as soon as the
last @var{n} tokens and groupings match a rule.  This is because such a
simple strategy is inadequate to handle most languages.  Instead, when a
reduction is possible, the parser sometimes ``looks ahead'' at the next
token in order to decide what to do.

When a token is read, it is not immediately shifted; first it becomes the
@dfn{look-ahead token}, which is not on the stack.  Now the parser can
perform one or more reductions of tokens and groupings on the stack, while
the look-ahead token remains off to the side.  When no more reductions
should take place, the look-ahead token is shifted onto the stack.  This
does not mean that all possible reductions have been done; depending on the
token type of the look-ahead token, some rules may choose to delay their
application.

Here is a simple case where look-ahead is needed.  These three rules define
expressions which contain binary addition operators and postfix unary
factorial operators (@samp{!}), and allow parentheses for grouping.

@example
expr:     term '+' expr
        | term
        ;

term:     '(' expr ')'
        | term '!'
        | NUMBER
        ;
@end example

Suppose that the tokens @w{@samp{1 + 2}} have been read and shifted; what
should be done?  If the following token is @samp{)}, then the first three
tokens must be reduced to form an @code{expr}.  This is the only valid
course, because shifting the @samp{)} would produce a sequence of symbols
@w{@code{term ')'}}, and no rule allows this.

If the following token is @samp{!}, then it must be shifted immediately so
that @w{@samp{2 !}} can be reduced to make a @code{term}.  If instead the
parser were to reduce before shifting, @w{@samp{1 + 2}} would become an
@code{expr}.  It would then be impossible to shift the @samp{!} because
doing so would produce on the stack the sequence of symbols @code{expr
'!'}.  No rule allows that sequence.

@vindex yychar
The current look-ahead token is stored in the variable @code{yychar}.
@xref{Action Features}.

@node Shift/Reduce, Precedence, Look-Ahead, Algorithm
@section Shift/Reduce Conflicts
@cindex conflicts
@cindex shift/reduce conflicts
@cindex dangling @code{else}
@cindex @code{else}, dangling

Suppose we are parsing a language which has if-then and if-then-else
statements, with a pair of rules like this:

@example
if_stmt:
          IF expr THEN stmt
        | IF expr THEN stmt ELSE stmt
        ;
@end example

@noindent
(Here we assume that @code{IF}, @code{THEN} and @code{ELSE} are
terminal symbols for specific keyword tokens.)

When the @code{ELSE} token is read and becomes the look-ahead token, the
contents of the stack (assuming the input is valid) are just right for
reduction by the first rule.  But it is also legitimate to shift the
@code{ELSE}, because that would lead to eventual reduction by the second
rule.

This situation, where either a shift or a reduction would be valid, is
called a @dfn{shift/reduce conflict}.  Bison is designed to resolve these
conflicts by choosing to shift, unless otherwise directed by operator
precedence declarations.  To see the reason for this, let's contrast
it with the other alternative.

Since the parser prefers to shift the @code{ELSE}, the result is to attach
the else-clause to the innermost if-statement, making these two inputs
equivalent:

@example
if x then if y then win(); else lose;

if x then do; if y then win(); else lose; end;
@end example

But if the parser chose to reduce when possible rather than shift, the
result would be to attach the else-clause to the outermost if-statement,
making these two inputs equivalent:

@example
if x then if y then win(); else lose;

if x then do; if y then win(); end; else lose;
@end example

The conflict exists because the grammar as written is ambiguous: either
parsing of the simple nested if-statement is legitimate.  The established
convention is that these ambiguities are resolved by attaching the
else-clause to the innermost if-statement; this is what Bison accomplishes
by choosing to shift rather than reduce.  (It would ideally be cleaner to
write an unambiguous grammar, but that is very hard to do in this case.)
This particular ambiguity was first encountered in the specifications of
Algol 60 and is called the ``dangling @code{else}'' ambiguity.

To avoid warnings from Bison about predictable, legitimate shift/reduce
conflicts, use the @code{%expect @var{n}} declaration.  There will be no
warning as long as the number of shift/reduce conflicts is exactly @var{n}.
@xref{Expect Decl}.

@node Precedence, Contextual Precedence, Shift/Reduce, Algorithm
@section Operator Precedence
@cindex operator precedence
@cindex precedence of operators

Another situation where shift/reduce conflicts appear is in arithmetic
expressions.  Here shifting is not always the preferred resolution; the
Bison declarations for operator precedence allow you to specify when to
shift and when to reduce.

@menu
* Why Precedence::      An example showing why precedence is needed.
* Using Precedence::    How to specify precedence in Bison grammars.
* Precedence Examples:: How these features are used in the previous example.
* How Precedence::      How they work.
@end menu

@node Why Precedence, Using Precedence, Precedence, Precedence
@subsection When Precedence is Needed

Consider the following ambiguous grammar fragment (ambiguous because the
input @w{@samp{1 - 2 * 3}} can be parsed in two different ways):

@example
expr:     expr '-' expr
        | expr '*' expr
        | expr '<' expr
        | '(' expr ')'
        @dots{}
        ;
@end example

@noindent
Suppose the parser has seen the tokens @samp{1}, @samp{-} and @samp{2};
should it reduce them via the rule for the addition operator?  It depends
on the next token.  Of course, if the next token is @samp{)}, we must
reduce; shifting is invalid because no single rule can reduce the token
sequence @w{@samp{- 2 )}} or anything starting with that.  But if the next
token is @samp{*} or @samp{<}, we have a choice: either shifting or
reduction would allow the parse to complete, but with different
results.

To decide which one Bison should do, we must consider the results.  If the
next operator token @var{op} is shifted, then it must be reduced first in
order to permit another opportunity to reduce the sum.  The result is (in
effect) @w{@samp{1 - (2 @var{op} 3)}}.  On the other hand, if the
subtraction is reduced before shifting @var{op}, the result is @w{@samp{(1
- 2) @var{op} 3}}.  Clearly, then, the choice of shift or reduce should
depend on the relative precedence of the operators @samp{-} and @var{op}:
@samp{*} should be shifted first, but not @samp{<}.

@cindex associativity
What about input like @w{@samp{1 - 2 - 5}}; should this be @w{@samp{(1 - 2)
- 5}} or @w{@samp{1 - (2 - 5)}}?  For most operators we prefer the former,
which is called @dfn{left association}.  The latter alternative, @dfn{right
association}, is desirable for assignment operators.  The choice of left or
right association is a matter of whether the parser chooses to shift or
reduce when the stack contains @w{@samp{1 - 2}} and the look-ahead token is
@samp{-}: shifting makes right-associativity.

@node Using Precedence, Precedence Examples, Why Precedence, Precedence
@subsection How to Specify Operator Precedence
@findex %left
@findex %right
@findex %nonassoc

Bison allows you to specify these choices with the operator precedence
declarations @code{%left} and @code{%right}.  Each such declaration
contains a list of tokens, which are operators whose precedence and
associativity is being declared.  The @code{%left} declaration makes all
those operators left-associative and the @code{%right} declaration makes
them right-associative.  A third alternative is @code{%nonassoc}, which
declares that it is a syntax error to find the same operator twice ``in a
row''.

The relative precedence of different operators is controlled by the order
in which they are declared.  The first @code{%left} or @code{%right}
declaration declares the operators whose precedence is lowest, the next
such declaration declares the operators whose precedence is a little
higher, and so on.

@node Precedence Examples, How Precedence, Using Precedence, Precedence
@subsection Precedence Examples

In our example, we would want the following declarations:

@example
%left '<'
%left '-'
%left '*'
@end example

In a more complete example, which supports other operators as well, we
would declare them in groups of equal precedence.  For example, @code{'+'} is
declared with @code{'-'}:

@example
%left '<' '>' '=' NE LE GE
%left '+' '-'
%left '*' '/'
@end example

@noindent
(Here @code{NE} and so on stand for the operators for ``not equal''
and so on.  We assume that these tokens are more than one character long
and therefore are represented by names, not character literals.)

@node How Precedence,, Precedence Examples, Precedence
@subsection How Precedence Works

The first effect of the precedence declarations is to assign precedence
levels to the terminal symbols declared.  The second effect is to assign
precedence levels to certain rules: each rule gets its precedence from the
last terminal symbol mentioned in the components.  (You can also specify
explicitly the precedence of a rule.  @xref{Contextual Precedence}.)

Finally, the resolution of conflicts works by comparing the precedence of
the rule being considered with that of the look-ahead token.  If the
token's precedence is higher, the choice is to shift.  If the rule's
precedence is higher, the choice is to reduce.  If they have equal
precedence, the choice is made based on the associativity of that
precedence level.  The verbose output file made by @samp{-v}
(@pxref{Invocation}) says how each conflict was resolved.

Not all rules and not all tokens have precedence.  If either the rule or
the look-ahead token has no precedence, then the default is to shift.

@node Contextual Precedence, Parser States, Precedence, Algorithm
@section Operators with Context-Dependent Precedence
@cindex context-dependent precedence
@cindex unary operator precedence
@findex %prec

Often the precedence of an operator depends on the context.  This sounds
outlandish at first, but it is really very common.  For example, a minus
sign typically has a very high precedence as a unary operator, and a
somewhat lower precedence (lower than multiplication) as a binary operator.

The Bison precedence declarations, @code{%left}, @code{%right} and
@code{%nonassoc}, can only be used once for a given token; so a token has
only one precedence declared in this way.  For context-dependent
precedence, you need to use an additional mechanism: the @code{%prec}
modifier for rules.@refill

The @code{%prec} modifier declares the precedence of a particular rule by
specifying a terminal symbol whose predecence should be used for that rule.
It's not necessary for that symbol to appear otherwise in the rule.  The
modifier's syntax is:

@example
%prec @var{terminal-symbol}
@end example

@noindent
and it is written after the components of the rule.  Its effect is to
assign the rule the precedence of @var{terminal-symbol}, overriding
the precedence that would be deduced for it in the ordinary way.  The
altered rule precedence then affects how conflicts involving that rule
are resolved (@pxref{Precedence}).

Here is how @code{%prec} solves the problem of unary minus.  First, declare
a precedence for a fictitious terminal symbol named @code{UMINUS}.  There
are no tokens of this type, but the symbol serves to stand for its
precedence:

@example
@dots{}
%left '+' '-'
%left '*'
%left UMINUS
@end example

Now the precedence of @code{UMINUS} can be used in specific rules:

@example
exp:    @dots{}
        | exp '-' exp
        @dots{}
        | '-' exp %prec UMINUS
@end example

@node Parser States, Reduce/Reduce, Contextual Precedence, Algorithm
@section Parser States
@cindex finite-state machine
@cindex parser state
@cindex state (of parser)

The function @code{yyparse} is implemented using a finite-state machine.
The values pushed on the parser stack are not simply token type codes; they
represent the entire sequence of terminal and nonterminal symbols at or
near the top of the stack.  The current state collects all the information
about previous input which is relevant to deciding what to do next.

Each time a look-ahead token is read, the current parser state together
with the type of look-ahead token are looked up in a table.  This table
entry can say, ``Shift the look-ahead token.''  In this case, it also
specifies the new parser state, which is pushed onto the top of the
parser stack.  Or it can say, ``Reduce using rule number @var{n}.''
This means that a certain of tokens or groupings are taken off the top
of the stack, and replaced by one grouping.  In other words, that number
of states are popped from the stack, and one new state is pushed.

There is one other alternative: the table can say that the look-ahead token
is erroneous in the current state.  This causes error processing to begin
(@pxref{Error Recovery}).

@node Reduce/Reduce,, Parser States, Algorithm
@section Reduce/Reduce conflicts
@cindex reduce/reduce conflict

A reduce/reduce conflict occurs if there are two or more rules that apply
to the same sequence of input.  This usually indicates a serious error
in the grammar.

For example, here is an erroneous attempt to define a sequence
of zero or more @code{word} groupings.

@example
sequence: /* empty */
                @{ printf ("empty sequence\n"); @}
        | word
                @{ printf ("single word %s\n", $1); @}
        | sequence word
                @{ printf ("added word %s\n", $2); @}
        ;
@end example

@noindent
The error is an ambiguity: there is more than one way to parse a single
@code{word} into a @code{sequence}.  It could be reduced directly via the
second rule.  Alternatively, nothing-at-all could be reduced into a
@code{sequence} via the first rule, and this could be combined with the
@code{word} using the third rule.

You might think that this is a distinction without a difference, because it
does not change whether any particular input is valid or not.  But it does
affect which actions are run.  One parsing order runs the second rule's
action; the other runs the first rule's action and the third rule's action.
In this example, the output of the program changes.

Bison resolves a reduce/reduce conflict by choosing to use the rule that
appears first in the grammar, but it is very risky to rely on this.  Every
reduce/reduce conflict must be studied and usually eliminated.  Here is the
proper way to define @code{sequence}:

@example
sequence: /* empty */
                @{ printf ("empty sequence\n"); @}
        | sequence word
                @{ printf ("added word %s\n", $2); @}
        ;
@end example

Here is another common error that yields a reduce/reduce conflict:

@example
sequence: /* empty */
        | sequence words
        | sequence redirects
        ;

words:    /* empty */
        | words word
        ;

redirects:/* empty */
        | redirects redirect
        ;
@end example

@noindent
The intention here is to define a sequence which can contain either
@code{word} or @code{redirect} groupings.  The individual definitions of
@code{sequence}, @code{words} and @code{redirects} are error-free, but the
three together make a subtle ambiguity: even an empty input can be parsed
in infinitely many ways!

Consider: nothing-at-all could be a @code{words}.  Or it could be two
@code{words} in a row, or three, or any number.  It could equally well be a
@code{redirects}, or two, or any number.  Or it could be a @code{words}
followed by three @code{redirects} and another @code{words}.  And so on.

Here are two ways to correct these rules.  First, to make it a single level
of sequence:

@example
sequence: /* empty */
        | sequence word
        | sequence redirect
        ;
@end example

Second, to prevent either a @code{words} or a @code{redirects}
from being empty:

@example
sequence: /* empty */
        | sequence words
        | sequence redirects
        ;

words:    word
        | words word
        ;

redirects:redirect
        | redirects redirect
        ;
@end example

@node Error Recovery, Context Dependency, Algorithm, Top
@chapter Error Recovery
@cindex error recovery
@cindex recovery from errors

It is not usually acceptable to have the program terminate on a parse
error.  For example, a compiler should recover sufficiently to parse the
rest of the input file and check it for errors; a calculator should accept
another expression.

In a simple interactive command parser where each input is one line, it may
be sufficient to allow @code{yyparse} to return 1 on error and have the
caller ignore the rest of the input line when that happens (and then call
@code{yyparse} again).  But this is inadequate for a compiler, because it
forgets all the syntactic context leading up to the error.  A syntax error
deep within a function in the compiler input should not cause the compiler
to treat the following line like the beginning of a source file.

@findex error
You can define how to recover from a syntax error by writing rules to
recognize the special token @code{error}.  This is a terminal symbol that
is always defined (you need not declare it) and reserved for error
handling.  The Bison parser generates an @code{error} token whenever a
syntax error happens; if you have provided a rule to recognize this token
in the current context, the parse can continue.  For example:

@example
stmnts:  /* empty string */
        | stmnts '\n'
        | stmnts exp '\n'
        | stmnts error '\n'
@end example

The fourth rule in this example says that an error followed by a newline
makes a valid addition to any @code{stmnts}.

What happens if a syntax error occurs in the middle of an @code{exp}?  The
error recovery rule, interpreted strictly, applies to the precise sequence
of a @code{stmnts}, an @code{error} and a newline.  If an error occurs in
the middle of an @code{exp}, there will probably be some additional tokens
and subexpressions on the stack after the last @code{stmnts}, and there
will be tokens to read before the next newline.  So the rule is not
applicable in the ordinary way.

But Bison can force the situation to fit the rule, by discarding part of
the semantic context and part of the input.  First it discards states and
objects from the stack until it gets back to a state in which the
@code{error} token is acceptable.  (This means that the subexpressions
already parsed are discarded, back to the last complete @code{stmnts}.)  At
this point the @code{error} token can be shifted.  Then, if the old
look-ahead token is not acceptable to be shifted next, the parser reads
tokens and discards them until it finds a token which is acceptable.  In
this example, Bison reads and discards input until the next newline
so that the fourth rule can apply.

The choice of error rules in the grammar is a choice of strategies for
error recovery.  A simple and useful strategy is simply to skip the rest of
the current input line or current statement if an error is detected:

@example
stmnt: error ';'  /* on error, skip until ';' is read */
@end example

It is also useful to recover to the matching close-delimiter of an
opening-delimiter that has already been parsed.  Otherwise the
close-delimiter will probably appear to be unmatched, and generate another,
spurious error message:

@example
primary:  '(' expr ')'
        | '(' error ')'
        @dots{}
        ;
@end example

Error recovery strategies are necessarily guesses.  When they guess wrong,
one syntax error often leads to another.  In the above example, the error
recovery rule guesses that an error is due to bad input within one
@code{stmnt}.  Suppose that instead a spurious semicolon is inserted in the
middle of a valid @code{stmnt}.  After the error recovery rule recovers
from the first error, another syntax error will be found straightaway,
since the text following the spurious semicolon is also an invalid
@code{stmnt}.

To prevent an outpouring of error messages, the parser will output no error
message for another syntax error that happens shortly after the first; only
after three consecutive input tokens have been successfully shifted will
error messages resume.

Note that rules which accept the @code{error} token may have actions, just
as any other rules can.

@findex yyerrok
You can make error messages resume immediately by using the macro
@code{yyerrok} in an action.  If you do this in the error rule's action, no
error messages will be suppressed.  This macro requires no arguments;
@samp{yyerrok;} is a valid C statement.

@findex yyclearin
The previous look-ahead token is reanalyzed immediately after an error.  If
this is unacceptable, then the macro @code{yyclearin} may be used to clear
this token.  Write the statement @samp{yyclearin;} in the error rule's
action.

For example, suppose that on a parse error, an error handling routine is
called that advances the input stream to some point where parsing should
once again commence.  The next symbol returned by the lexical scanner is
probably correct.  The previous look-ahead token ought to be discarded
with @samp{yyclearin;}.

@node Context Dependency, Debugging, Error Recovery, Top
@chapter Handling Context Dependencies

The Bison paradigm is to parse tokens first, then group them into larger
syntactic units.  In many languages, the meaning of a token is affected by
its context.  Although this violates the Bison paradigm, certain techniques
(known as @dfn{kludges}) may enable you to write Bison parsers for such
languages.

@menu
* Semantic Tokens::     Token parsing can depend on the semantic context.
* Lexical Tie-ins::     Token parsing can depend on the syntactic context.
* Tie-in Recovery::     Lexical tie-ins have implications for how
			  error recovery rules must be written.
@end menu

(Actually, ``kludge'' means any technique that gets its job done but is
neither clean nor robust.)

@node Semantic Tokens, Lexical Tie-ins, Context Dependency, Context Dependency
@section Semantic Info in Token Types

The C language has a context dependency: the way an identifier is used
depends on what its current meaning is.  For example, consider this:

@example
foo (x);
@end example

This looks like a function call statement, but if @code{foo} is a typedef
name, then this is actually a declaration of @code{x}.  How can a Bison
parser for C decide how to parse this input?

The method used in GNU C is to have two different token types,
@code{IDENTIFIER} and @code{TYPENAME}.  When @code{yylex} finds an
identifier, it looks up the current declaration of the identifier in order
to decide which token type to return: @code{TYPENAME} if the identifier is
declared as a typedef, @code{IDENTIFIER} otherwise.

The grammar rules can then express the context dependency by the choice of
token type to recognize.  @code{IDENTIFIER} is accepted as an expression,
but @code{TYPENAME} is not.  @code{TYPENAME} can start a declaration, but
@code{IDENTIFIER} cannot.  In contexts where the meaning of the identifier
is @emph{not} significant, such as in declarations that can shadow a
typedef name, either @code{TYPENAME} or @code{IDENTIFIER} is
accepted---there is one rule for each of the two token types.

This technique is simple to use if the decision of which kinds of
identifiers to allow is made at a place close to where the identifier is
parsed.  But in C this is not always so: C allows a declaration to
redeclare a typedef name provided an explicit type has been specified
earlier:

@example
typedef int foo, bar, lose;
static foo (bar);        /* @r{redeclare @code{bar} as static variable} */
static int foo (lose);   /* @r{redeclare @code{foo} as function} */
@end example

Unfortunately, the name being declared is separated from the declaration
construct itself by a complicated syntactic structure---the ``declarator''.

As a result, the part of Bison parser for C needs to be duplicated, with
all the nonterminal names changed: once for parsing a declaration in which
a typedef name can be redefined, and once for parsing a declaration in
which that can't be done.  Here is a part of the duplication, with actions
omitted for brevity:

@example
initdcl:
          declarator maybeasm '='
          init
        | declarator maybeasm
        ;

notype_initdcl:
          notype_declarator maybeasm '='
          init
        | notype_declarator maybeasm
        ;
@end example

@noindent
Here @code{initdcl} can redeclare a typedef name, but @code{notype_initdcl}
cannot.  The distinction between @code{declarator} and
@code{notype_declarator} is the same sort of thing.

There is some similarity between this technique and a lexical tie-in
(described next), in that information which alters the lexical analysis is
changed during parsing by other parts of the program.  The difference is
here the information is global, and is used for other purposes in the
program.  A true lexical tie-in has a special-purpose flag controlled by
the syntactic context.

@node Lexical Tie-ins, Tie-in Recovery, Semantic Tokens, Context Dependency
@section Lexical Tie-ins
@cindex lexical tie-in

One way to handle context-dependency is the @dfn{lexical tie-in}: a flag
which is set by Bison actions, whose purpose is to alter the way tokens are
parsed.

For example, suppose we have a language vaguely like C, but with a special
construct @samp{hex (@var{hex-expr})}.  After the keyword @code{hex} comes
an expression in parentheses in which all integers are hexadecimal.  In
particular, the token @samp{a1b} must be treated as an integer rather than
as an identifier if it appears in that context.  Here is how you can do it:

@example
%@{
int hexflag;
%@}
%%
@dots{}
expr:   IDENTIFIER
        | constant
	| HEX '('
                @{ hexflag = 1; @}
          expr ')'
		@{ hexflag = 0;
                   $$ = $4; @}
        | expr '+' expr
                @{ $$ = make_sum ($1, $3); @}
        @dots{}
        ;

constant:
          INTEGER
        | STRING
        ;
@end example

@noindent
Here we assume that @code{yylex} looks at the value of @code{hexflag}; when
it is nonzero, all integers are parsed in hexadecimal, and tokens starting
with letters are parsed as integers if possible.

The declaration of @code{hexflag} shown in the C declarations section of
the parser file is needed to make it accessible to the actions (@pxref{C
Declarations}).  You must also write the code in @code{yylex} to obey the
flag.

@node Tie-in Recovery,, Lexical Tie-ins, Context Dependency
@section Lexical Tie-ins and Error Recovery

Lexical tie-ins make strict demands on any error recovery rules you have.
@xref{Error Recovery}.

The reason for this is that the purpose of an error recovery rule is to
abort the parsing of one construct and resume in some larger construct.
For example, in C-like languages, a typical error recovery rule is to skip
tokens until the next semicolon, and then start a new statement, like this:

@example
stmt:   expr ';'
        | IF '(' expr ')' stmt @{ @dots{} @}
        @dots{}
        error ';'
                @{ hexflag = 0; @}
        ;
@end example

If there is a syntax error in the middle of a @samp{hex (@var{expr})}
construct, this error rule will apply, and then the action for the
completed @samp{hex (@var{expr})} will never run.  So @code{hexflag} would
remain set for the entire rest of the input, or until the next @code{hex}
keyword, causing identifiers to be misinterpreted as integers.

To avoid this problem the error recovery rule itself clears @code{hexflag}.

There may also be an error recovery rule that works within expressions.
For example, there could be a rule which applies within parentheses
and skips to the close-parenthesis:

@example
expr:   @dots{}
        | '(' expr ')'
                @{ $$ = $2; @}
        | '(' error ')'
        @dots{}
@end example

If this rule acts within the @code{hex} construct, it is not going to abort
that construct (since it applies to an inner level of parentheses within
the construct).  Therefore, it should not clear the flag: the rest of
the @code{hex} construct should be parsed with the flag still in effect.

What if there is an error recovery rule which might abort out of the
@code{hex} construct or might not, depending on circumstances?  There is no
way you can write the action to determine whether a @code{hex} construct is
being aborted or not.  So if you are using a lexical tie-in, you had better
make sure your error recovery rules are not of this kind.  Each rule must
be such that you can be sure that it always will, or always won't, have to
clear the flag.

@node Debugging, Invocation, Context Dependency, Top
@chapter Debugging Your Parser
@findex YYDEBUG
@findex yydebug
@cindex debugging
@cindex tracing the parser

If a Bison grammar compiles properly but doesn't do what you want when it
runs, the @code{yydebug} parser-trace feature can help you figure out why.

To enable compilation of trace facilities, you must define the macro
@code{YYDEBUG} when you compile the parser.  You could use @samp{-DYYDEBUG}
as a compiler option or you could put @samp{#define YYDEBUG} in the C
declarations section of the grammar file (@pxref{C Declarations}).
Alternatively, use the @samp{-t} option when you run Bison
(@pxref{Invocation}).  I always define @code{YYDEBUG} so that debugging is
always possible.

The trace facility uses @code{stderr}, so you must add @w{@code{#include
<stdio.h>}} to the C declarations section unless it is already there.

Once you have compiled the program with trace facilities, the way to
request a trace is to store a nonzero value in the variable @code{yydebug}.
You can do this by making the C code do it (in @code{main}, perhaps), or
you can alter the value with a C debugger.

Each step taken by the parser when @code{yydebug} is nonzero produces a
line or two of trace information, written on @code{stderr}.  The trace
messages tell you these things:

@itemize @bullet
@item
Each time the parser calls @code{yylex}, what kind of token was read.

@item
Each time a token is shifted, the depth and complete contents of the
state stack (@pxref{Parser States}).

@item
Each time a rule is reduced, which rule it is, and the complete contents
of the state stack afterward.
@end itemize

To make sense of this information, it helps to refer to the listing file
produced by the Bison @samp{-v} option (@pxref{Invocation}).  This file
shows the meaning of each state in terms of positions in various rules, and
also what each state will do with each possible input token.  As you read
the successive trace messages, you can see that the parser is functioning
according to its specification in the listing file.  Eventually you will
arrive at the place where something undesirable happens, and you will see
which parts of the grammar are to blame.

The parser file is a C program and you can use C debuggers on it, but it's
not easy to interpret what it is doing.  The parser function is a
finite-state machine interpreter, and aside from the actions it executes
the same code over and over.  Only the values of variables show where in
the grammar it is working.

@node Invocation, Table of Symbols, Debugging, Top
@chapter Invocation of Bison; Command Options
@cindex invoking Bison
@cindex Bison invocation
@cindex options for Bison invocation

The usual way to invoke Bison is as follows:

@example
bison @var{infile}
@end example

Here @var{infile} is the grammar file name, which usually ends in
@samp{.y}.  The parser file's name is made by replacing the @samp{.y}
with @samp{.tab.c}.  Thus, @samp{bison foo.y} outputs
@file{foo.tab.c}.@refill

These options can be used with Bison:

@table @samp
@item -d
Write an extra output file containing macro definitions for the token
type names defined in the grammar and the semantic value type
@code{YYSTYPE}, as well as a few @code{extern} variable declarations.

If the parser output file is named @file{@var{name}.c} then this file
is named @file{@var{name}.h}.@refill

This output file is essential if you wish to put the definition of
@code{yylex} in a separate source file, because @code{yylex} needs to
be able to refer to token type codes and the variable
@code{yylval}.  @xref{Token Values}.@refill

@item -l
Don't put any @code{#line} preprocessor commands in the parser file.
Ordinarily Bison puts them in the parser file so that the C compiler
and debuggers will associate errors with your source file, the
grammar file.  This option causes them to associate errors with the
parser file, treating it an independent source file in its own right.

@item -o @var{outfile}
Specify the name @var{outfile} for the parser file.

The other output files' names are constructed from @var{outfile}
as described under the @samp{-v} and @samp{-d} switches.

@item -t
Output a definition of the macro @code{YYDEBUG} into the parser file,
so that the debugging facilities are compiled.  @xref{Debugging}.

@item -v
Write an extra output file containing verbose descriptions of the
parser states and what is done for each type of look-ahead token in
that state.

This file also describes all the conflicts, both those resolved by
operator precedence and the unresolved ones.

The file's name is made by removing @samp{.tab.c} or @samp{.c} from
the parser output file name, and adding @samp{.output} instead.@refill

Therefore, if the input file is @file{foo.y}, then the parser file is
called @file{foo.tab.c} by default.  As a consequence, the verbose
output file is called @file{foo.output}.@refill

@item -y
Equivalent to @samp{-o y.tab.c}; the parser output file is called
@file{y.tab.c}, and the other outputs are called @file{y.output} and
@file{y.tab.h}.  The purpose of this switch is to imitate Yacc's
output file name conventions.@refill
@end table

@node Table of Symbols, Glossary, Invocation, Top
@appendix Table of Bison Symbols
@cindex Bison symbols, table of
@cindex symbols in Bison, table of

@table @code
@item error
A token name reserved for error recovery.  This token may be used in
grammar rules so as to allow the Bison parser to recognize an error in
the grammar without halting the process.  In effect, a sentence
containing an error may be recognized as valid.  On a parse error, the
token @code{error} becomes the current look-ahead token.  Actions
corresponding to @code{error} are then executed, and the look-ahead
token is reset to the token that originally caused the violation.
@xref{Error Recovery}.

@item YYABORT
Macro to pretend that an unrecoverable syntax error has occurred, by
making @code{yyparse} return 1 immediately.  The error reporting
function @code{yyerror} is not called.  @xref{Parser Function}.

@item YYACCEPT
Macro to pretend that a complete utterance of the language has been
read, by making @code{yyparse} return 0 immediately.  @xref{Parser
Function}.

@item YYERROR
Macro to pretend that a syntax error has just been detected: call
@code{yyerror} and then perform normal error recovery if possible
(@pxref{Error Recovery}), or (if recovery is impossible) make
@code{yyparse} return 1.  @xref{Error Recovery}.

@item YYLTYPE
Macro for the data type of @code{yylloc}; a structure with four
members.  @xref{Token Positions}.

@item YYSTYPE
Macro for the data type of semantic values; @code{int} by default.
@xref{Value Type}.

@item yychar
External integer variable that contains the integer value of the
current look-ahead token.  (In a pure parser, it is a local variable
within @code{yyparse}.)  Error-recovery rule actions may examine this
variable.  @xref{Action Features}.

@item yyclearin
Macro used in error-recovery rule actions.  It clears the previous
look-ahead token.  @xref{Error Recovery}.

@item yydebug
External integer variable set to zero by default.  If @code{yydebug}
is given a nonzero value, the parser will output information on input
symbols and parser action.  @xref{Debugging}.

@item yyerrok
Macro to cause parser to recover immediately to its normal mode
after a parse error.  @xref{Error Recovery}.

@item yyerror
User-supplied function to be called by @code{yyparse} on error.  The
function receives one argument, a pointer to a character string
containing an error message.  @xref{Error Reporting}.

@item yylex
User-supplied lexical analyzer function, called with no arguments
to get the next token.  @xref{Lexical}.

@item yylval
External variable in which @code{yylex} should place the semantic
value associated with a token.  (In a pure parser, it is a local
variable within @code{yyparse}, and its address is passed to
@code{yylex}.)  @xref{Token Values}.

@item yylloc
External variable in which @code{yylex} should place the line and
column numbers associated with a token.  (In a pure parser, it is a
local variable within @code{yyparse}, and its address is passed to
@code{yylex}.)  You can ignore this variable if you don't use the
@samp{@@} feature in the grammar actions.  @xref{Token Positions}.

@item yynerrs
Global variable which Bison increments each time there is a parse
error.  (In a pure parser, it is a local variable within
@code{yyparse}.)  @xref{Error Reporting}.

@item yyparse
The parser function produced by Bison; call this function to start
parsing.  @xref{Parser Function}.

@item %left
Bison declaration to assign left associativity to token(s).
@xref{Precedence Decl}.

@item %nonassoc
Bison declaration to assign nonassociativity to token(s).
@xref{Precedence Decl}.

@item %prec
Bison declaration to assign a precedence to a specific rule.
@xref{Contextual Precedence}.

@item %pure_parser
Bison declaration to request a pure (reentrant) parser.
@xref{Pure Decl}.

@item %right
Bison declaration to assign right associativity to token(s).
@xref{Precedence Decl}.

@item %start
Bison declaration to specify the start symbol.  @xref{Start Decl}.

@item %token
Bison declaration to declare token(s) without specifying precedence.
@xref{Token Decl}.

@item %type
Bison declaration to declare nonterminals.  @xref{Type Decl}.

@item %union
Bison declaration to specify several possible data types for semantic
values.  @xref{Union Decl}.
@end table

These are the punctuation and delimiters used in Bison input:

@table @samp
@item %%
Delimiter used to separate the grammar rule section from the
Bison declarations section or the additional C code section.
@xref{Grammar Layout}.

@item %@{ %@}
All code listed between @samp{%@{} and @samp{%@}} is copied directly
to the output file uninterpreted.  Such code forms the ``C
declarations'' section of the input file.  @xref{Grammar Outline}.

@item /*@dots{}*/
Comment delimiters, as in C.

@item :
Separates a rule's result from its components.  @xref{Rules}.

@item ;
Terminates a rule.  @xref{Rules}.

@item |
Separates alternate rules for the same result nonterminal.
@xref{Rules}.
@end table

@node Glossary, Index, Table of Symbols, top
@appendix Glossary
@cindex glossary

@table @asis
@item Backus-Naur Form (BNF)
Formal method of specifying context-free grammars.  BNF was first used
in the @cite{ALGOL-60} report, 1963.  @xref{Language and Grammar}.

@item Context-free grammars
Grammars specified as rules that can be applied regardless of context.
Thus, if there is a rule which says that an integer can be used as an
expression, integers are allowed @emph{anywhere} an expression is
permitted.  @xref{Language and Grammar}.

@item Dynamic allocation
Allocation of memory that occurs during execution, rather than at
compile time or on entry to a function.

@item Empty string
Analogous to the empty set in set theory, the empty string is a
character string of length zero.

@item Finite-state stack machine
A ``machine'' that has discrete states in which it is said to exist at
each instant in time.  As input to the machine is processed, the
machine moves from state to state as specified by the logic of the
machine.  In the case of the parser, the input is the language being
parsed, and the states correspond to various stages in the grammar
rules.  @xref{Algorithm}.

@item Grouping
A language construct that is (in general) grammatically divisible;
for example, `expression' or `declaration' in C.  @xref{Language and
Grammar}.

@item Infix operator
An arithmetic operator that is placed between the operands on which it
performs some operation.

@item Input stream
A continuous flow of data between devices or programs.

@item Language construct
One of the typical usage schemas of the language.  For example, one of
the constructs of the C language is the @code{if} statement.
@xref{Language and Grammar}.

@item Left associativity
Operators having left associativity are analyzed from left to right:
@samp{a+b+c} first computes @samp{a+b} and then combines with
@samp{c}.  @xref{Precedence}.

@item Left recursion
A rule whose result symbol is also its first component symbol;
for example, @samp{expseq1 : expseq1 ',' exp;}.  @xref{Recursion}.

@item Left-to-right parsing
Parsing a sentence of a language by analyzing it token by token from
left to right.  @xref{Algorithm}.

@item Lexical analyzer (scanner)
A function that reads an input stream and returns tokens one by one.
@xref{Lexical}.

@item Lexical tie-in
A flag, set by actions in the grammar rules, which alters the way
tokens are parsed.  @xref{Lexical Tie-ins}.

@item Look-ahead token
A token already read but not yet shifted.  @xref{Look-Ahead}.

@item Nonterminal symbol
A grammar symbol standing for a grammatical construct that can
be expressed through rules in terms of smaller constructs; in other
words, a construct that is not a token.  @xref{Symbols}.

@item Parse error
An error encountered during parsing of an input stream due to invalid
syntax.  @xref{Error Recovery}.

@item Parser
A function that recognizes valid sentences of a language by analyzing
the syntax structure of a set of tokens passed to it from a lexical
analyzer.

@item Postfix operator
An arithmetic operator that is placed after the operands upon which it
performs some operation.

@item Reduction
Replacing a string of nonterminals and/or terminals with a single
nonterminal, according to a grammar rule.  @xref{Algorithm}.

@item Reentrant
A reentrant subprogram is a subprogram which can be in invoked any
number of times in parallel, without interference between the various
invocations.  @xref{Pure Decl}.

@item Reverse polish notation
A language in which all operators are postfix operators.

@item Right recursion
A rule whose result symbol is also its last component symbol;
for example, @samp{expseq1: exp ',' expseq1;}.  @xref{Recursion}.

@item Semantics
In computer languages the semantics are specified by the actions
taken for each instance of the language, i.e., the meaning of
each statement.  @xref{Semantics}.

@item Shift
A parser is said to shift when it makes the choice of analyzing
further input from the stream rather than reducing immediately some
already-recognized rule.  @xref{Algorithm}.

@item Single-character literal
A single character that is recognized and interpreted as is.
@xref{Grammar in Bison}.

@item Start symbol
The nonterminal symbol that stands for a complete valid utterance in
the language being parsed.  The start symbol is usually listed as the
first nonterminal symbol in a language specification.  @xref{Start
Decl}.

@item Symbol table
A data structure where symbol names and associated data are stored
during parsing to allow for recognition and use of existing
information in repeated uses of a symbol.  @xref{Multi-function Calc}.

@item Token
A basic, grammatically indivisible unit of a language.  The symbol
that describes a token in the grammar is a terminal symbol.
The input of the Bison parser is a stream of tokens which comes from
the lexical analyzer.  @xref{Symbols}.

@item Terminal symbol
A grammar symbol that has no rules in the grammar and therefore
is grammatically indivisible.  The piece of text it represents
is a token.  @xref{Language and Grammar}.
@end table

@node Index, , Glossary, top
@unnumbered Index

@printindex cp

@contents

@bye
